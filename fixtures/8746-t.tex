\documentclass[oneside]{book}
\usepackage[francais,english]{babel}
\selectlanguage{english}
\begin{document}

\thispagestyle{empty}
\small
\begin{verbatim}
Project Gutenberg's History of Modern Mathematics, by David Eugene Smith

Copyright laws are changing all over the world. Be sure to check the
copyright laws for your country before downloading or redistributing
this or any other Project Gutenberg eBook.

This header should be the first thing seen when viewing this Project
Gutenberg file.  Please do not remove it.  Do not change or edit the
header without written permission.

Please read the "legal small print," and other information about the
eBook and Project Gutenberg at the bottom of this file.  Included is
important information about your specific rights and restrictions in
how the file may be used.  You can also find out about how to make a
donation to Project Gutenberg, and how to get involved.

**Welcome To The World of Free Plain Vanilla Electronic Texts**

**eBooks Readable By Both Humans and By Computers, Since 1971**

*****These eBooks Were Prepared By Thousands of Volunteers!*****


Title: History of Modern Mathematics
       Mathematical Monographs No. 1

Author: David Eugene Smith

Release Date: August, 2005 [EBook #8746]
[Yes, we are more than one year ahead of schedule]
[This file was first posted on August 9, 2003]

Edition: 10

Language: English

Character set encoding: ASCII / TeX

*** START OF THE PROJECT GUTENBERG EBOOK HISTORY OF MODERN MATHEMATICS ***


Produced by David Starner, John Hagerson,
and the Online Distributed Proofreading Team

\end{verbatim}

\normalsize
\newpage



\renewcommand{\chaptername}{Article}

\frontmatter



\begin{center}
\noindent
MATHEMATICAL MONOGRAPHS

\bigskip
EDITED BY MANSFIELD MERRIMAN AND ROBERT S. WOODWARD

\bigskip\bigskip\huge
No. 1

\bigskip
HISTORY OF MODERN MATHEMATICS.

\bigskip\large
BY

\bigskip
DAVID EUGENE SMITH,

\bigskip\footnotesize\textsc{
PROFESSOR OF MATHEMATICS IN TEACHERS COLLEGE, COLUMBIA UNIVERSITY.}

\bigskip

FOURTH EDITION, ENLARGED.

1906

\end{center}
\newpage

\fbox{\parbox{\columnwidth}{\textbf{MATHEMATICAL MONOGRAPHS.}\newline
\small\textsc{edited by}\normalsize\newline
\textbf{Mansfield Merriman and Robert S. Woodward.}

\bigskip
\textbf{No. 1. HISTORY OF MODERN MATHEMATICS.}\newline
By \textsc{David Eugene Smith.}

\bigskip
\textbf{No. 2. SYNTHETIC PROJECTIVE GEOMETRY.}\newline
By \textsc{George Bruce Halsted.}

\bigskip
\textbf{No. 3. DETERMINANTS.}\newline
By \textsc{Laenas Gifford Weld.}

\bigskip
\textbf{No. 4. HYPERBOLIC FUNCTIONS.}\newline
By \textsc{James McMahon.}

\bigskip
\textbf{No. 5. HARMONIC FUNCTIONS.}\newline
By \textsc{William E. Byerly.}

\bigskip
\textbf{No. 6. GRASSMANN'S SPACE ANALYSIS.}\newline
By \textsc{Edward W. Hyde.}

\bigskip
\textbf{No. 7. PROBABILITY AND THEORY OF ERRORS.}\newline
By \textsc{Robert S. Woodward.}

\bigskip
\textbf{No. 8. VECTOR ANALYSIS AND QUATERNIONS.}\newline
By \textsc{Alexander Macfarlane.}

\bigskip
\textbf{No. 9. DIFFERENTIAL EQUATIONS.}\newline
By \textsc{William Woolsey Johnson.}

\bigskip
\textbf{No. 10. THE SOLUTION OF EQUATIONS.}\newline
By \textsc{Mansfield Merriman.}

\bigskip
\textbf{No. 11. FUNCTIONS OF A COMPLEX VARIABLE.}\newline
By \textsc{Thomas S. Fiske.}

\normalsize
} }
\indent

\newpage

\chapter{EDITORS' PREFACE.}

The volume called Higher Mathematics, the first edition of which was
published in 1896, contained eleven chapters by eleven authors, each
chapter being independent of the others, but all supposing the
reader to have at least a mathematical training equivalent to that
given in classical and engineering colleges. The publication of that
volume is now discontinued and the chapters are issued in separate
form. In these reissues it will generally be found that the
monographs are enlarged by additional articles or appendices which
either amplify the former presentation or record recent
advances. This plan of publication has been arranged in order to
meet the demand of teachers and the convenience of classes, but it
is also thought that it may prove advantageous to readers in special
lines of mathematical literature.

It is the intention of the publishers and editors to add other
monographs to the series from time to time, if the call for the same
seems to warrant it. Among the topics which are under consideration
are those of elliptic functions, the theory of numbers, the group
theory, the calculus of variations, and non-Euclidean geometry;
possibly also monographs on branches of astronomy, mechanics, and
mathematical physics may be included. It is the hope of the editors
that this form of publication may tend to promote mathematical study
and research over a wider field than that which the former volume
has occupied.

\bigskip

December, 1905.

\normalsize

\chapter{AUTHOR'S PREFACE.}

This little work was published about ten years ago as a chapter in
Merriman and Woodward's Higher Mathematics. It was written before
the numerous surveys of the development of science in the past
hundred years, which appeared at the close of the nineteenth
century, and it therefore had more reason for being then than now,
save as it can now call attention, to these later contributions. The
conditions under which it was published limited it to such a small
compass that it could do no more than present a list of the most
prominent names in connection with a few important topics. Since it
is necessary to use the same plates in this edition, simply adding a
few new pages, the body of the work remains substantially as it
first appeared. The book therefore makes no claim to being history,
but stands simply as an outline of the prominent movements in
mathematics, presenting a few of the leading names, and calling
attention to some of the bibliography of the subject.

It need hardly be said that the field of mathematics is now so
extensive that no one can longer pretend to cover it, least of all
the specialist in any one department. Furthermore it takes a century
or more to weigh men and their discoveries, thus making the
judgment of contemporaries often quite worthless. In spite of these
facts, however, it is hoped that these pages will serve a good
purpose by offering a point of departure to students desiring to
investigate the movements of the past hundred years. The
bibliography in the foot-notes and in Articles 19 and 20 will serve
at least to open the door, and this in itself is a sufficient excuse
for a work of this nature.

\textsc{Teachers College, Columbia University,}

December, 1905.

\normalsize

\tableofcontents

%% CONTENTS.

%% ART.

%% 1. INTRODUCTION

%% 2. THEORY OF NUMBERS

%% 3. IRRATIONAL AND TRANSCENDENT NUMBERS

%% 4. COMPLEX NUMBERS

%% 5. QUATERNIONS AND AUSDEHNUNGSLEHRE

%% 6. THEORY OF EQUATIONS

%% 7. SUBSTITUTIONS AND GROUPS

%% 8. DETERMINANTS

%% 9. QUANTICS

%% 10. CALCULUS

%% 11. DIFFERENTIAL EQUATIONS

%% 12. INFINITE SERIES

%% 13. THEORY OF FUNCTIONS

%% 14. PROBABILITIES AND LEAST SQUARES

%% 15. ANALYTIC GEOMETRY

%% 16. MODERN GEOMETRY

%% 17. TRIGONOMETRY AND ELEMENTARY GEOMETRY

%% 18. NON-EUCLIDEAN GEOMETRY

%% 19. BIBLIOGRAPHY

%% 20. GENERAL TENDENCIES

%% INDEX

\mainmatter

\chapter{INTRODUCTION.}

In considering the history of modern mathematics two questions at
once arise: (1) what limitations shall be placed upon the term
Mathematics; (2) what force shall be assigned to the word Modern? In
other words, how shall Modern Mathematics be defined?

In these pages the term Mathematics will be limited to the domain of
pure science. Questions of the applications of the various branches
will be considered only incidentally. Such great contributions as
those of Newton in the realm of mathematical physics, of Laplace in
celestial mechanics, of Lagrange and Cauchy in the wave theory, and
of Poisson, Fourier, and Bessel in the theory of heat, belong rather
to the field of applications.

In particular, in the domain of numbers reference will be made to
certain of the contributions to the general theory, to the men who
have placed the study of irrational and transcendent numbers upon a
scientific foundation, and to those who have developed the modern
theory of complex numbers and its elaboration in the field of
quaternions and Ausdehnungslehre. In the theory of equations the
names of some of the leading investigators will be mentioned,
together with a brief statement of the results which they
secured. The impossibility of solving the quintic will lead to a
consideration of the names of the founders of the group theory and
of the doctrine of determinants. This phase of higher algebra will
be followed by the theory of forms, or quantics. The later
development of the calculus, leading to differential equations and
the theory of functions, will complete the algebraic side, save for
a brief reference to the theory of probabilities. In the domain of
geometry some of the contributors to the later development of the
analytic and synthetic fields will be mentioned, together with the
most noteworthy results of their labors. Had the author's space not
been so strictly limited he would have given lists of those who have
worked in other important lines, but the topics considered have been
thought to have the best right to prominent place under any
reasonable definition of Mathematics.

Modern Mathematics is a term by no means well defined. Algebra
cannot be called modern, and yet the theory of equations has
received some of its most important additions during the nineteenth
century, while the theory of forms is a recent creation. Similarly
with elementary geometry; the labors of Lobachevsky and Bolyai
during the second quarter of the century threw a new light upon the
whole subject, and more recently the study of the triangle has added
another chapter to the theory. Thus the history of modern
mathematics must also be the modern history of ancient branches,
while subjects which seem the product of late generations have root
in other centuries than the present.

How unsatisfactory must be so brief a sketch may be inferred from a
glance at the Index du Rep\'ertoire Bibliographique des Sciences
Math\'ematiques (Paris, 1893), whose seventy-one pages contain the
mere enumeration of subjects in large part modern, or from a
consideration of the twenty-six volumes of the Jahrbuch \"uber die
Fortschritte der Mathematik, which now devotes over a thousand pages
a year to a record of the progress of the science.\footnote{The
foot-notes give only a few of the authorities which might easily be
cited. They are thought to include those from which considerable
extracts have been made, the necessary condensation of these
extracts making any other form of acknowledgment impossible.}

The seventeenth and eighteenth centuries laid the foundations of
much of the subject as known to-day. The discovery of the analytic
geometry by Descartes, the contributions to the theory of numbers by
Fermat, to algebra by Harriot, to geometry and to mathematical
physics by Pascal, and the discovery of the differential calculus by
Newton and Leibniz, all contributed to make the seventeenth century
memorable. The eighteenth century was naturally one of great
activity. Euler and the Bernoulli family in Switzerland,
d'Alembert, Lagrange, and Laplace in Paris, and Lambert in Germany,
popularized Newton's great discovery, and extended both its theory
and its applications. Accompanying this activity, however, was a too
implicit faith in the calculus and in the inherited principles of
mathematics, which left the foundations insecure and necessitated
their strengthening by the succeeding generation.

The nineteenth century has been a period of intense study of first
principles, of the recognition of necessary limitations of various
branches, of a great spread of mathematical knowledge, and of the
opening of extensive fields for applied mathematics. Especially
influential has been the establishment of scientific schools and
journals and university chairs. The great renaissance of geometry is
not a little due to the foundation of the \'Ecole Polytechnique in
Paris (1794-5), and the similar schools in Prague (1806), Vienna
(1815), Berlin (1820), Karlsruhe (1825), and numerous other
cities. About the middle of the century these schools began to exert
a still a greater influence through the custom of calling to them
mathematicians of high repute, thus making Z\"urich, Karlsruhe,
Munich, Dresden, and other cities well known as mathematical centers.

In 1796 appeared the first number of the Journal de l'\'Ecole
Polytechnique. Crelle's Journal f\"ur die reine und angewandte
Mathematik appeared in 1826, and ten years later Liouville began the
publication of the Journal de Math\'ematiques pures et appliqu\'ees,
which has been continued by Resal and Jordan. The Cambridge
Mathematical Journal was established in 1839, and merged into the
Cambridge and Dublin Mathematical Journal in 1846. Of the other
periodicals which have contributed to the spread of mathematical
knowledge, only a few can be mentioned: the Nouvelles Annales de
Math\'ematiques (1842), Grunert's Archiv der Mathematik (1843),
Tortolini's Annali di Scienze Matematiche e Fisiche (1850),
Schl\"omilch's Zeitschrift f\"ur Mathematik und Physik (1856), the
Quarterly Journal of Mathematics (1857), Battaglini's Giornale di
Matematiche (1863), the Mathematische Annalen (1869), the Bulletin
des Sciences Math\'ematiques (1870), the American Journal of
Mathematics (1878), the Acta Mathematica (1882), and the Annals of
Mathematics (1884).\footnote{For a list of current mathematical
journals see the Jahrbuch \"uber die Fortschritte der Mathematik. A
small but convenient list of standard periodicals is given in Carr's
Synopsis of Pure Mathematics, p. 843; Mackay, J. S., Notice sur le
journalisme math\'ematique en Angleterre, Association fran\c{c}aise
pour l'Avancement des Sciences, 1893, II, 303; Cajori, F., Teaching
and History of Mathematics in the United States, pp. 94, 277;
Hart, D.~S., History of American Mathematical Periodicals, The Analyst,
Vol. II, p. 131.} To this list should be added a recent venture,
unique in its aims, namely, L'Interm\'ediaire des Math\'ematiciens
(1894), and two annual publications of great value, the Jahrbuch
already mentioned (1868), and the Jahresbericht der deutschen
Math\-e\-ma\-tik\-er-Vereinigung (1892).
%% Are those the correct hyphenation points?

To the influence of the schools and the journals must be added that
of the various learned societies\footnote{For a list of such
societies consult any recent number of the Philosophical
Transactions of Royal Society of London. Dyck, W., Einleitung zu dem
f\"ur den mathematischen Teil der deutschen
Universit\"atsausstellung ausgegebenen Specialkatalog, Mathematical
Papers Chicago Congress (New York, 1896), p. 41.} whose published
proceedings are widely known, together with the increasing
liberality of such societies in the preparation of complete works of
a monumental character.

The study of first principles, already mentioned, was a natural
consequence of the reckless application of the new calculus and the
Cartesian geometry during the eighteenth century. This development
is seen in theorems relating to infinite series, in the fundamental
principles of number, rational, irrational, and complex, and in the
concepts of limit, contiunity, function, the infinite, and the
infinitesimal. But the nineteenth century has done more than
this. It has created new and extensive branches of an importance
which promises much for pure and applied mathematics. Foremost among
these branches stands the theory of functions founded by Cauchy,
Riemann, and Weierstrass, followed by the descriptive and
projective geometries, and the theories of groups, of forms, and of
determinants.

The nineteenth century has naturally been one of specialization. At
its opening one might have hoped to fairly compass the mathematical,
physical, and astronomical sciences, as did Lagrange, Laplace, and
Gauss. But the advent of the new generation, with Monge and Carnot,
Poncelet and Steiner, Galois, Abel, and Jacobi, tended to split
mathematics into branches between which the relations were long to
remain obscure. In this respect recent years have seen a reaction,
the unifying tendency again becoming prominent through the theories
of functions and groups.\footnote{Klein, F., The Present State of
Mathematics, Mathematical Papers of Chicago Congress (New York,
1896), p. 133.}

\chapter{THEORY OF NUMBERS.}

The Theory of Numbers,\footnote{Cantor, M., Geschichte der
Mathematik, Vol. III, p. 94; Smith, H.~J.~S., Report on the theory
of numbers; Collected Papers, Vol. I; Stolz, O., Gr\"ossen und
Zahien, Leipzig. 1891.} a favorite study among the Greeks, had its
renaissance in the sixteenth and seventeenth centuries in the labors
of Viete, Bachet de Meziriac, and especially Fermat. In the
eighteenth century Euler and Lagrange contributed to the theory, and
at its close the subject began to take scientific form through the
great labors of Legendre (1798), and Gauss (1801). With the latter's
Disquisitiones Arithmetic\ae (1801) may be said to begin the modern
theory of numbers. This theory separates into two branches, the one
dealing with integers, and concerning itself especially with (1) the
study of primes, of congruences, and of residues, and in particular
with the law of reciprocity, and (2) the theory of forms, and the
other dealing with complex numbers.

The Theory of Primes\footnote{Brocard, H., Sur la fr\'equence et la
totalit\'e des nombres premiers; Nouvelle Correspondence de
Math\'ematiques, Vols. V and VI; gives recent history to 1879.} has
attracted many investigators during the nineteenth century, but
the results have been detailed rather than general. Tch\'e\-bi\-chef
(1850)
% Another arbitary hyphenation point
was the first to reach any valuable conclusions in the way of
ascertaining the number of primes between two given limits. Riemann
(1859) also gave a well-known formula for the limit of the number of
primes not exceeding a given number.

The Theory of Congruences may be said to start with Gauss's
Disquisitiones. He introduced the symbolism $a \equiv b \pmod c$,
and explored most of the field. Tch\'ebichef published in
1847 a work in Russian upon the subject, and in France Serret has
done much to make the theory known.

Besides summarizing the labors of his predecessors in the theory of
numbers, and adding many original and noteworthy contributions, to
Legendre may be assigned the fundamental theorem which bears his
name, the Law of Reciprocity of Quadratic Residues. This law,
discovered by induction and enunciated by Euler, was first proved by
Legendre in his Th\'eorie des Nombres (1798) for special
cases. Independently of Euler and Legendre, Gauss discovered the law
about 1795, and was the first to give a general proof. To the
subject have also contributed Cauchy, perhaps the most versatile of
French mathematicians of the century; Dirichlet, whose Vorlesungen
\"uber Zahlentheorie, edited by Dedekind, is a classic; Jacobi,
who introduced the generalized symbol which bears his name;
Liouville, Zeller, Eisenstein, Kummer, and Kronecker. The theory has
been extended to include cubic and biquadratic reciprocity, notably
by Gauss, by Jacobi, who first proved the law of cubic reciprocity,
and by Kummer.

To Gauss is also due the representation of numbers by binary
quadratic forms. Cauchy, Poinsot (1845), Lebesque (1859, 1868), and
notably Hermite have added to the subject. In the theory of ternary
forms Eisenstein has been a leader, and to him and H.~J.~S.~Smith is
also due a noteworthy advance in the theory of forms in
general. Smith gave a complete classification of ternary quadratic
forms, and extended Gauss's researches concerning real quadratic
forms to complex forms. The investigations concerning the
representation of numbers by the sum of 4, 5, 6, 7, 8 squares were
advanced by Eisenstein and the theory was completed by Smith.

In Germany, Dirichlet was one of the most zealous workers in the
theory of numbers, and was the first to lecture upon the subject in
a German university. Among his contributions is the extension of
Fermat's theorem on $x^n+y^n=z^n$, which Euler and Legendre had proved
for $n$ = 3, 4, Dirichlet showing that $x^5+y^5 \neq az^5$. Among
the later French writers are Borel; Poincar\'e, whose memoirs are
numerous and valuable; Tannery, and Stieltjes. Among the leading
contributors in Germany are Kronecker, Kummer, Schering, Bachmann,
and Dedekind. In Austria Stolz's Vorlesungen \"uber allgemeine
Arithmetik (1885-86), and in England Mathews' Theory of Numbers
(Part I, 1892) are among the most scholarly of general works.
Genocchi, Sylvester, and J.~W.~L.~Glaisher have also added to the
theory.

\chapter{IRRATIONAL AND TRANSCENDENT NUMBERS.}

The sixteenth century saw the final acceptance of negative numbers,
integral and fractional. The seventeenth century saw decimal
fractions with the modern notation quite generally used by
mathematicians. The next hundred years saw the imaginary become a
powerful tool in the hands of De Moivre, and especially of
Euler. For the nineteenth century it remained to complete the
theory of complex numbers, to separate irrationals into algebraic
and transcendent, to prove the existence of transcendent numbers,
and to make a scientific study of a subject which had remained
almost dormant since Euclid, the theory of irrationals. The year
1872 saw the publication of the theories of Weierstrass (by his
pupil Kossak), Heine (Crelle, 74), G. Cantor (Annalen, 5), and
Dedekind. M\'eray had taken in 1869 the same point of departure as
Heine, but the theory is generally referred to the year
1872. Weierstrass's method has been completely set forth by
Pincherle (1880), and Dedekind's has received additional prominence
through the author's later work (1888) and the recent indorsement by
Tannery (1894). Weierstrass, Cantor, and Heine base their theories
on infinite series, while Dedekind founds his on the idea of a cut
(Schnitt) in the system of real numbers, separating all rational
numbers into two groups having certain characteristic
properties. The subject has received later contributions at the
hands of Weierstrass, Kronecker (Crelle, 101), and M\'eray.

Continued Fractions, closely related to irrational numbers {and due
to Ca\-tal\-di, 1613),\footnote{But see Favaro, A., Notizie storiche
sulle frazioni continue dal secolo decimoterzo al decimosettimo,
Boncompagni's Bulletino, Vol. VII, 1874, pp. 451, 533.} received
attention at the hands of Euler, and at the opening of the
nineteenth century were brought into prominence through the writings
of Lagrange. Other noteworthy contributions have been made by
Druckenm\"uller (1837), Kunze (1857), Lemke (1870), and G\"unther
(1872). Ramus (1855) first connected the subject with determinants,
resulting, with the subsequent contributions of Heine, M\"obius, and
G\"unther, in the theory of Kettenbruchdeterminanten. Dirichlet
also added to the general theory, as have numerous contributors to
the applications of the subject.

Transcendent Numbers\footnote{Klein, F., Vortr\"age \"uber
ausgew\"ahlte Fragen der Elementargeometrie, 1895, p. 38; Bachmann,
P., Vorlesungen \"uber die Natur der Irrationalzahlen, 1892.} were
first distinguished from algebraic irrationals by
Kronecker. Lambert proved (1761) that $\pi$ cannot be rational, and
that $e^n$ ($n$ being a rational number) is irrational, a proof,
however, which left much to be desired. Legendre (1794) completed
Lambert's proof, and showed that $\pi$ is not the square root of a
rational number. Liouville (1840) showed that neither $e$ nor
$e^2$ can be a root of an integral quadratic equation. But the
existence of transcendent numbers was first established by Liouville
(1844, 1851), the proof being subsequently displaced by G. Cantor's
(1873). Hermite (1873) first proved $e$ transcendent, and Lindemann
(1882), starting from Hermite's conclusions, showed the same for
$\pi$. Lindemann's proof was much simplified by Weierstrass (1885),
still further by Hilbert (1893), and has finally been made
elementary by Hurwitz and Gordan.

\chapter{COMPLEX NUMBERS.}

The Theory of Complex Numbers\footnote{Riecke, F., Die Rechnung mit
Richtungszahlen, 1856, p. 161; Hankel, H., Theorie der komplexen
Zahlensysteme, Leipzig, 1867; Holzm\"uller, G., Theorie der
isogonalen Verwandtschaften, 1882, p. 21; Macfarlane, A., The
Imaginary of Algebra, Proceedings of American Association 1892,
p. 33; Baltzer, R., Einf\"uhrung der komplexen Zahlen, Crelle, 1882;
Stolz, O., Vorlesungen \"uber allgemeine Arithmetik, 2. Theil,
Leipzig, 1886.} may be said to have attracted attention as early as
the sixteenth century in the recognition, by the Italian
algebraists, of imaginary or impossible roots. In the seventeenth
century Descartes distinguished between real and imaginary roots,
and the eighteenth saw the labors of De Moivre and Euler. To De
Moivre is due (1730) the well-known formula which bears his name,
$(\cos \theta + i \sin
\theta)^{n} = \cos n \theta + i \sin n \theta$, and to Euler (1748)
the formula $\cos \theta + i \sin \theta = e ^{\theta i}$.

The geometric notion of complex quantity now arose, and as a result
the theory of complex numbers received a notable expansion. The idea
of the graphic representation of complex numbers had appeared,
however, as early as 1685, in Wallis's De Algebra tractatus. In the
eighteenth century K\"uhn (1750) and Wessel (about 1795) made
decided advances towards the present theory. Wessel's memoir
appeared in the Proceedings of the Copenhagen Academy for 1799, and
is exceedingly clear and complete, even in comparison with modern
works. He also considers the sphere, and gives a quaternion theory
from which he develops a complete spherical trigonometry. In 1804
the Abb\'e Bu\'ee independently came upon the same idea which Wallis
had suggested, that $\pm\sqrt{-1}$ should represent a unit line, and
its negative, perpendicular to the real axis. Bu\'ee's paper was
not published until 1806, in which year Argand also issued a
pamphlet on the same subject. It is to Argand's essay that the
scientific foundation for the graphic representation of complex
numbers is now generally referred. Nevertheless, in 1831 Gauss
found the theory quite unknown, and in 1832 published his chief
memoir on the subject, thus bringing it prominently before the
mathematical world. Mention should also be made of an excellent
little treatise by Mourey (1828), in which the foundations for the
theory of directional numbers are scientifically laid. The general
acceptance of the theory is not a little due to the labors of Cauchy
and Abel, and especially the latter, who was the first to boldly use
complex numbers with a success that is well known.

The common terms used in the theory are chiefly due to the
founders. Argand called $\cos \phi + i \sin \phi$ the ``direction
factor'', and $r = \sqrt{a^2+b^2}$ the ``modulus''; Cauchy (1828)
called $\cos \phi + i \sin \phi$ the ``reduced form'' (l'expression
r\'eduite); Gauss used $i$ for $\sqrt{-1}$, introduced the term
``complex number'' for $a+bi$, and called $a^2+b^2$ the ``norm.'' The
expression ``direction coefficient'', often used for $\cos \phi + i
\sin \phi$, is due to Hankel (1867), and ``absolute value,'' for
``modulus,'' is due to Weierstrass.

Following Cauchy and Gauss have come a number of contributors of
high rank, of whom the following may be especially mentioned: Kummer
(1844), Kronecker (1845), Scheffler (1845, 1851, 1880), Bellavitis
(1835, 1852), Peacock (1845), and De Morgan (1849). M\"obius must
also be mentioned for his numerous memoirs on the geometric
applications of complex numbers, and Dirichlet for the expansion of
the theory to include primes, congruences, reciprocity, etc., as in
the case of real numbers.

Other types\footnote{Chapman, C. H., Weierstrass and Dedekind on
General Complex Numbers, in Bulletin New York Mathematical Society,
Vol. I, p. 150; Study, E., Aeltere und neuere Untersuchungen \"uber
Systeme complexer Zahlen, Mathematical Papers Chicago Congress,
p. 367; bibliography, p. 381.} have been studied, besides the
familiar $a + bi$, in which $i$ is the root of $x^2 + 1 = 0$. Thus
Eisenstein has studied the type $a + bj$, $j$ being a complex root
of $x^3 - 1 = 0$. Similarly, complex types have been derived from
$x^k - 1 = 0$ ($k$ prime). This generalization is largely due to
Kummer, to whom is also due the theory of Ideal
Numbers,\footnote{Klein, F., Evanston Lectures, Lect. VIII.} which
has recently been simplified by Klein (1893) from the point of view
of geometry. A further complex theory is due to Galois, the basis
being the imaginary roots of an irreducible congruence, $F(x) \equiv 0$
(mod $p$, a prime). The late writers (from 1884) on the general
theory include Weierstrass, Schwarz, Dedekind, H\"older, Berloty,
Poincar\'e, Study, and Macfarlane.

\chapter{QUATERNIONS AND AUSDEHNUNGSLEHRE.}

Quaternions and Ausdehnungslehre\footnote{Tait, P.~G., on
Quaternions, Encyclop\ae{}dia Britannica; Schlegel, V., Die
Grassmann'sche Ausdehnungslehre, Schl\"omilch's Zeitschrift,
Vol. XLI.} are so closely related to complex quantity, and the
latter to complex number, that the brief sketch of their development
is introduced at this point. Caspar Wessel's contributions to the
theory of complex quantity and quaternions remained unnoticed in
the proceedings of the Copenhagen Academy. Argand's attempts to
extend his method of complex numbers beyond the space of two
dimensions failed. Servois (1813), however, almost trespassed on the
quaternion field. Nevertheless there were fewer traces of the theory
anterior to the labors of Hamilton than is usual in the case of
great discoveries. Hamilton discovered the principle of quaternions
in 1843, and the next year his first contribution to the theory


appeared, thus extending the Argand idea to three-dimensional
space. This step necessitated an expansion of the idea of $r(\cos
\phi + j \sin \phi)$ such that while $r$ should be a real number and
$\phi$ a real angle, $i$, $j$, or $k$ should be any directed unit
line such that $i^2 = j^2 = k^2 = -1$. It also necessitated a
withdrawal of the commutative law of multiplication, the adherence
to which obstructed earlier discovery. It was not until 1853 that
Hamilton's Lectures on Quarternions appeared, followed (1866) by his
Elements of Quaternions.

In the same year in which Hamilton published his discovery (1844),
Grassmann gave to the world his famous work, Die lineale
Ausdehnungslehre, although he seems to have been in possession of
the theory as early as 1840. Differing from Hamilton's Quaternions
in many features, there are several essential principles held in
common which each writer discovered independently of the
other.\footnote{These are set forth in a paper by J.~W.~Gibbs,
Nature, Vol. XLIV, p. 79.}

Following Hamilton, there have appeared in Great Britain numerous
papers and works by Tait (1867), Kelland and Tait (1873), Sylvester,
and McAulay (1893). On the Continent Hankel (1867), Ho\"uel (1874),
and Laisant (1877, 1881) have written on the theory, but it has
attracted relatively little attention. In America, Benjamin Peirce
(1870) has been especially prominent in developing the quaternion
theory, and Hardy (1881), Macfarlane, and Hathaway (1896) have
contributed to the subject. The difficulties have been largely in
the notation. In attempting to improve this symbolism Macfarlane has
aimed at showing how a space analysis can be developed embracing
algebra, trigonometry, complex numbers, Grassmann's method, and
quaternions, and has considered the general principles of vector and
versor analysis, the versor being circular, elliptic logarithmic, or
hyperbolic. Other recent contributors to the algebra of vectors are
Gibbs (from 1881) and Heaviside (from 1885).

The followers of Grassmann\footnote{For bibliography see Schlegel,
V., Die Grassmann'sche Ausdehnungslehre, Schl\"omilch's Zeitschrift,
Vol. XLI.} have not been much more numerous than those of
Hamilton. Schlegel has been one of the chief contributors in
Germany, and Peano in Italy. In America, Hyde (Directional Calculus,
1890) has made a plea for the Grassmann theory.\footnote{For
Macfarlane's Digest of views of English and American writers, see
Proceedings American Association for Advancement of Science, 1891.}

Along lines analogous to those of Hamilton and Grassmann have been
the contributions of Scheffler. While the two former sacrificed the
commutative law, Scheffler (1846, 1851, 1880) sacrificed the
distributive. This sacrifice of fundamental laws has led to an
investigation of the field in which these laws are valid, an
investigation to which Grassmann (1872), Cayley, Ellis, Boole,
Schr\"oder (1890-91), and Kraft (1893) have contributed. Another
great contribution of Cayley's along similar lines is the theory of
matrices (1858).

\chapter{THEORY OF EQUATIONS.}

The Theory of Numerical Equations\footnote{Cayley, A., Equations,
and Kelland, P., Algebra, in Encyclop\ae{}dia Britannica; Favaro, A.,
Notizie storico-critiche sulla costruzione delle equazioni. Modena,
1878; Cantor, M., Geschichte der Mathematik, Vol. III, p. 375.}
concerns itself first with the location of the roots, and then with
their approximation. Neither problem is new, but the first
noteworthy contribution to the former in the nineteenth century was
Budan's (1807). Fourier's work was undertaken at about the same
time, but appeared posthumously in 1831. All processes were,
however, exceedingly cumbersome until Sturm (1829) communicated to
the French Academy the famous theorem which bears his name and which
constitutes one of the most brilliant discoveries of algebraic
analysis.

The Approximation of the Roots, once they are located, can be made
by several processes. Newton (1711), for example, gave a method
which Fourier perfected; and Lagrange (1767) discovered an ingenious
way of expressing the root as a continued fraction, a process which
Vincent (1836) elaborated. It was, however, reserved for Horner
(1819) to suggest the most practical method yet known, the one now
commonly used. With Horner and Sturm this branch practically
closes. The calculation of the imaginary roots by approximation is
still an open field.

The Fundamental Theorem\footnote{Loria, Gino, Esame di alcune
ricerche concernenti l'esistenza di radici nelle equazioni
algebriche; Bibliotheca Mathematica, 1891, p. 99; bibliography on
p. 107. Pierpont, J., On the Ruffini-Abelian theorem, Bulletin of
American Mathematical Society, Vol. II, p. 200.} that every
numerical equation has a root was generally assumed until the latter
part of the eighteenth century. D'Alembert (1746) gave a
demonstration, as did Lagrange (1772), Laplace (1795), Gauss (1799)
and Argand (1806). The general theorem that every algebraic equation
of the $n$th degree has exactly $n$ roots and no more follows as a
special case of Cauchy's proposition (1831) as to the number of
roots within a given contour. Proofs are also due to Gauss, Serret,
Clifford (1876), Malet (1878), and many others.

The Impossibility of Expressing the Roots of an equation as
algebraic functions of the coefficients when the degree exceeds 4
was anticipated by Gauss and announced by Ruffini, and the belief in
the fact became strengthened by the failure of Lagrange's methods
for these cases. But the first strict proof is due to Abel, whose
early death cut short his labors in this and other fields.

The Quintic Equation has naturally been an object of special
study. Lagrange showed that its solution depends on that of a
sextic, ``Lagrange's resolvent sextic,'' and Malfatti and
Vandermonde investigated the construction of resolvents. The
resolvent sextic was somewhat simplified by Cockle and Harley
(1858-59) and by Cayley (1861), but Kronecker (1858) was the first
to establish a resolvent by which a real simplification was
effected. The transformation of the general quintic into the
trinomial form $x^5+ax+b=0$ by the extraction of square and cube
roots only, was first shown to be possible by Bring (1786) and
independently by Jerrard\footnote{Harley, R., A contribution of
the history \ldots of the general equation of the fifth degree,
Quarterly Journal of Mathematics, Vol. VI, p. 38.} (1834). Hermite
(1858) actually effected this reduction, by means of Tschirnhausen's
theorem, in connection with his solution by elliptic functions.

The Modern Theory of Equations may be said to date from Abel and
Galois. The latter's special memoir on the subject, not published
until 1846, fifteen years after his death, placed the theory on a
definite base. To him is due the discovery that to each equation
corresponds a group of substitutions (the ``group of the equation'')
in which are reflected its essential characteristics.\footnote{See
Art. 7.} Galois's untimely death left without sufficient
demonstration several important propositions, a gap which Betti
(1852) has filled. Jordan, Hermite, and Kronecker were also among
the earlier ones to add to the theory. Just prior to Galois's
researches Abel (1824), proceeding from the fact that a rational
function of five letters having less than five values cannot have
more than two, showed that the roots of a general quintic equation
cannot be expressed in terms of its coefficients by means of
radicals. He then investigated special forms of quintic equations
which admit of solution by the extraction of a finite number of
roots. Hermite, Sylvester, and Brioschi have applied the invariant
theory of binary forms to the same subject.

From the point of view of the group the solution by radicals,
formerly the goal of the algebraist, now appears as a single link in
a long chain of questions relative to the transformation of
irrationals and to their classification. Klein (1884) has handled
the whole subject of the quintic equation in a simple manner by
introducing the icosahedron equation as the normal form, and has
shown that the method can be generalized so as to embrace the whole
theory of higher equations.\footnote{Klein, F., Vorlesungen \"uber
das Ikosaeder, 1884.} He and Gordan (from 1879) have attacked those
equations of the sixth and seventh degrees which have a Galois group
of 168 substitutions, Gordan performing the reduction of the
equation of the seventh degree to the ternary problem. Klein (1888)
has shown that the equation of the twenty-seventh degree occurring
in the theory of cubic surfaces can be reduced to a normal problem
in four variables, and Burkhardt (1893) has performed the reduction,
the quaternary groups involved having been discussed by Maschke
(from 1887).

Thus the attempt to solve the quintic equation by means of radicals
has given place to their treatment by transcendents. Hermite (1858)
has shown the possibility of the solution, by the use of elliptic
functions, of any Bring quintic, and hence of any equation of the
fifth degree. Kronecker (1858), working from a different standpoint,
has reached the same results, and his method has since been
simplified by Brioschi. More recently Kronecker, Gordan, Kiepert,
and Klein, have contributed to the same subject, and the sextic
equation has been attacked by Maschke and Brioschi through the
medium of hyperelliptic functions.

Binomial Equations, reducible to the form $x^n - 1 = 0$, admit of
ready solution by the familiar trigonometric formula $x =
\cos\frac{2k\pi}{n} + i\sin\frac{2k\pi}{n}$; but it was reserved for
Gauss (1801) to show that an algebraic solution is
possible. Lagrange (1808) extended the theory, and its application
to geometry is one of the leading additions of the century. Abel,
generalizing Gauss's results, contributed the important theorem that
if two roots of an irreducible equation are so connected that the
one can be expressed rationally in terms of the other, the equation
yields to radicals if the degree is prime and otherwise depends on
the solution of lower equations. The binomial equation, or rather
the equation $\sum_0^{n-1} x^m = 0$, is one of this class
considered by Abel, and hence called (by Kronecker) Abelian
Equations. The binomial equation has been treated notably by
Richelot (1832), Jacobi (1837), Eisenstein (1844, 1850), Cayley
(1851), and Kronecker (1854), and is the subject of a treatise by
Bachmann (1872). Among the most recent writers on Abelian equations
is Pellet (1891).

Certain special equations of importance in geometry have been the
subject of study by Hesse, Steiner, Cayley, Clebsch, Salmon, and
Kummer. Such are equations of the ninth degree determining the
points of inflection of a curve of the third degree, and of the
twenty-seventh degree determining the points in which a curve of the
third degree can have contact of the fifth order with a conic.

Symmetric Functions of the coefficients, and those which remain
unchanged through some or all of the permutations of the roots, are
subjects of great importance in the present theory. The first
formulas for the computation of the symmetric functions of the
roots of an equation seem to have been worked out by Newton,
although Girard (1629) had given, without proof, a formula for the
power sum. In the eighteenth century Lagrange (1768) and Waring
(1770, 1782) contributed to the theory, but the first tables,
reaching to the tenth degree, appeared in 1809 in the Meyer-Hirsch
Aufgabensammlung. In Cauchy's celebrated memoir on determinants
(1812) the subject began to assume new prominence, and both he and
Gauss (1816) made numerous and valuable contributions to the
theory. It is, however, since the discoveries by Galois that the
subject has become one of great importance. Cayley (1857) has given
simple rules for the degree and weight of symmetric functions, and
he and Brioschi have simplified the computation of tables.

Methods of Elimination and of finding the resultant (Bezout) or
eliminant (De Morgan) occupied a number of eighteenth-century
algebraists, prominent among them being Euler (1748), whose method,
based on symmetric functions, was improved by Cramer (1750) and
Bezout (1764). The leading steps in the development are represented
by Lagrange (1770-71), Jacobi, Sylvester (1840), Cayley (1848,
1857), Hesse (1843, 1859), Bruno (1859), and Katter
(1876). Sylvester's dialytic method appeared in 1841, and to him is
also due (1851) the name and a portion of the theory of the
discriminant. Among recent writers on the general theory may be
mentioned Burnside and Pellet (from 1887).

\chapter{SUBSTITUTIONS AND GROUPS.}

The Theories of Substitutions and Groups\footnote{Netto, E., Theory
of Substitutions, translated by Cole; Cayley, A., Equations,
Encyclop\ae{}dia Britannica, 9th edition.} are among the most important
in the whole mathematical field, the study of groups and the search
for invariants now occupying the attention of many
mathematicians. The first recognition of the importance of the
combinatory analysis occurs in the problem of forming an
$m$th-degree equation having for roots $m$ of the roots of a given
$n$th-degree equation ($m < n$). For simple cases the problem goes
back to Hudde (1659). Saunderson (1740) noted that the determination
of the quadratic factors of a biquadratic expression necessarily
leads to a sextic equation, and Le S\oe{}ur (1748) and Waring (1762
to 1782) still further elaborated the idea.

Lagrange\footnote{Pierpont, James, Lagrange's Place in the Theory
of Substitutions, Bulletin of American Mathematical Society, Vol. I, p.
196.} first undertook a scientific treatment of the theory of
substitutions. Prior to his time the various methods of solving
lower equations had existed rather as isolated artifices than as
unified theory.\footnote{Matthiessen, L. Grundz\"uge der antiken
und modernen Algebra der litteralen Gleichungen, Leipzig, 1878.}
Through the great power of analysis possessed by Lagrange (1770,
1771) a common foundation was discovered, and on this was built the
theory of substitutions. He undertook to examine the methods then
known, and to show a priori why these succeeded below the quintic,
but otherwise failed. In his investigation he discovered the
important fact that the roots of all resolvents (r\'solvantes,
r\'eduites) which he examined are rational functions of the roots
of the respective equations. To study the properties of these
functions he invented a ``Calcul des Combinaisons.'' the first
important step towards a theory of substitutions. Mention should
also be made of the contemporary labors of Vandermonde (1770) as
foreshadowing the coming theory.

The next great step was taken by Ruffini\footnote{Burkhardt, H.,
Die Anf\"ange der Gruppentheorie und Paolo Ruffini, Abhandlungen zur
Geschichte der Mathematik, VI, 1892, p. 119. Italian by E. Pascal,
Brioschi's Annali di Matematica, 1894.} (1799). Beginning like
Lagrange with a discussion of the methods of solving lower
equations, he attempted the proof of the impossibility of solving
the quintic and higher equations. While the attempt failed, it is
noteworthy in that it opens with the classification of the various
``permutations'' of the coefficients, using the word to mean what
Cauchy calls a ``syst\`eme des substitutions conjugu\'ees,'' or
simply a ``syst\`eme conjugu\'e,'' and Galois calls a ``group of
substitutions.'' Ruffini distinguishes what are now called
intransitive, transitive and imprimitive, and transitive and
primitive groups, and (1801) freely uses the group of an equation
under the name ``l'assieme della permutazioni.'' He also publishes a
letter from Abbati to himself, in which the group idea is prominent.

To Galois, however, the honor of establishing the theory of groups
is generally awarded. He found that if $r_1, r_2, \ldots r_n$ are
the $n$ roots of an equation, there is always a group of
permutations of the $r$'s such that (1) every function of the roots
invariable by the substitutions of the group is rationally known,
and (2), reciprocally, every rationally determinable function of the
roots is invariable by the substitutions of the group. Galois also
contributed to the theory of modular equations and to that of
elliptic functions. His first publication on the group theory was
made at the age of eighteen (1829), but his contributions attracted
little attention until the publication of his collected papers in
1846 (Liouville, Vol. XI).

Cayley and Cauchy were among the first to appreciate the importance
of the theory, and to the latter especially are due a number of
important theorems. The popularizing of the subject is largely due
to Serret, who has devoted section IV of his algebra to the theory;
to Camille Jordan, whose Trait\'e des Substitutions is a classic;
and to Netto (1882), whose work has been translated into English by
Cole (1892). Bertrand, Hermite, Frobenius, Kronecker, and Mathieu
have added to the theory. The general problem to determine the
number of groups of $n$ given letters still awaits solution.

But overshadowing all others in recent years in carrying on the
labors of Galois and his followers in the study of discontinuous
groups stand Klein, Lie, Poincar\'e, and Picard. Besides these
discontinuous groups there are other classes, one of which, that of
finite continuous groups, is especially important in the theory of
differential equations. It is this class which Lie (from 1884) has
studied, creating the most important of the recent departments of
mathematics, the theory of transformation groups. Of value, too,
have been the labors of Killing on the structure of groups, Study's
application of the group theory to complex numbers, and the work of
Schur and Maurer.

\chapter{DETERMINANTS.}

The Theory of Determinants\footnote{Muir, T., Theory of Determinants
in the Historical Order of its Development, Part I, 1890; Baltzer,
R., Theorie und Anwendung der Determinanten. 1881. The writer is
under obligations to Professor Weld, who contributes Chap. II, for
valuable assistance in compiling this article.} may be said to take
its origin with Leibniz (1693), following whom Cramer (1750) added
slightly to the theory, treating the subject, as did his
predecessor, wholly in relation to sets of equations. The recurrent
law was first announced by Bezout (1764). But it was Vandermonde
(1771) who first recognized determinants as independent
functions. To him is due the first connected exposition of the
theory, and he may be called its formal founder. Laplace (1772)
gave the general method of expanding a determinant in terms of its
complementary minors, although Vandermonde had already given a
special case. Immediately following, Lagrange (1773) treated
determinants of the second and third order, possibly stopping here
because the idea of hyperspace was not then in vogue. Although
contributing nothing to the general theory, Lagrange was the first
to apply determinants to questions foreign to eliminations, and to
him are due many special identities which have since been brought
under well-known theorems. During the next quarter of a century
little of importance was done. Hindenburg (1784) and Rothe (1800)
kept the subject open, but Gauss (1801) made the next advance. Like
Lagrange, he made much use of determinants in the theory of
numbers. He introduced the word ``determinants'' (Laplace had used
``resultant''), though not in the present
signification,\footnote{``Numerum $bb-ac$, cuius indole
proprietates form\ae $(a, b, c)$ imprimis pendere in sequentibus
docebimus, determinantem huius uocabimus.''} but
rather as applied to the discriminant of a
quantic. Gauss also arrived at the notion of reciprocal
determinants, and came very near the multiplication theorem. The
next contributor of importance is Binet (1811, 1812), who formally
stated the theorem relating to the product of two matrices of $m$
columns and $n$ rows, which for the special case of $m = n$ reduces
to the multiplication theorem. On the same day (Nov. 30, 1812) that
Binet presented his paper to the Academy, Cauchy also presented one
on the subject. In this he used the word ``determinant'' in its
present sense, summarized and simplified what was then known on the
subject, improved the notation, and gave the multiplication theorem
with a proof more satisfactory than Binet's. He was the first to
grasp the subject as a whole; before him there were determinants,
with him begins their theory in its generality.

The next great contributor, and the greatest save Cauchy, was Jacobi
(from 1827). With him the word ``determinant'' received its final
acceptance. He early used the functional determinant which Sylvester
has called the ``Jacobian,'' and in his famous memoirs in Crelle for
1841 he specially treats this subject, as well as that class of
alternating functions which Sylvester has called ``Alternants.'' But
about the time of Jacobi's closing memoirs, Sylvester (1839) and
Cayley began their great work, a work which it is impossible to
briefly summarize, but which represents the development of the
theory to the present time.

The study of special forms of determinants has been the natural
result of the completion of the general theory. Axi-symmetric
determinants have been studied by Lebesgue, Hesse, and Sylvester;
per-symmetric determinants by Sylvester and Hankel; circulants by
Catalan, Spottiswoode, Glaisher, and Scott; skew determinants and
Pfaffians, in connection with the theory of orthogonal
transformation, by Cayley; continuants by Sylvester; Wronskians (so
called by Muir) by Christoffel and Frobenius; compound determinants
by Sylvester, Reiss, and Picquet; Jacobians and Hessians by
Sylvester; and symmetric gauche determinants by Trudi. Of the
text-books on the subject Spottiswoode's was the first. In America,
Hanus (1886) and Weld (1893) have published treatises.

\chapter{QUANTICS.}

The Theory of Qualities or Forms\footnote{Meyer, W.~F., Bericht
\"uber den gegenw\"artigen Stand der Invariantentheorie.
Jahresbericht der deutschen Mathematiker-Vereinigung, Vol. I,
1890-91; Berlin 1892, p. 97. See also the review by Franklin in
Bulletin New York Mathematical Society, Vol. III, p. 187; Biography
of Cayley, Collected Papers, VIII, p. ix, and Proceedings of Royal
Society, 1895.} appeared in embryo in the Berlin memoirs of Lagrange
(1773, 1775), who considered binary quadratic forms of the type
$ax^2+bxy+cy^2$, and established the invariance of the discriminant
of that type when $x+\lambda y$ is put for $x$. He classified forms
of that type according to the sign of $b^2-4ac$, and introduced the
ideas of transformation and equivalence. Gauss\footnote{See
Art. 2.} (1801) next took up the subject, proved the invariance of
the discriminants of binary and ternary quadratic forms, and
systematized the theory of binary quadratic forms, a subject
elaborated by H.~J.~S.~Smith, Eisenstein, Dirichlet, Lipschitz,
Poincar\'e, and Cayley. Galois also entered the field, in his
theory of groups (1829), and the first step towards the
establishment of the distinct theory is sometimes attributed to
Hesse in his investigations of the plane curve of the third order.

It is, however, to Boole (1841) that the real foundation of the
theory of invariants is generally ascribed. He first showed the
generality of the invariant property of the discriminant, which
Lagrange and Gauss had found for special forms. Inspired by Boole's
discovery Cayley took up the study in a memoir ``On the Theory of
Linear Transformations'' (1845), which was followed (1846) by
investigations concerning covariants and by the discovery of the
symbolic method of finding invariants. By reason of these
discoveries concerning invariants and covariants (which at first he
called ``hyperdeterminants'') he is regarded as the founder of what
is variously called Modern Algebra, Theory of Forms, Theory of
Quantics, and the Theory of Invariants and Covariants. His ten
memoirs on the subject began in 1854, and rank among the greatest
which have ever been produced upon a single theory. Sylvester soon
joined Cayley in this work, and his originality and vigor in
discovery soon made both himself and the subject prominent. To him
are due (1851-54) the foundations of the general theory, upon which
later writers have largely built, as well as most of the terminology
of the subject.

Meanwhile in Germany Eisenstein (1843) had become aware of the
simplest invariants and covariants of a cubic and biquadratic form,
and Hesse and Grassmann had both (1844) touched upon the
subject. But it was Aronhold (1849) who first made the new theory
known. He devised the symbolic method now common in Germany,
discovered the invariants of a ternary cubic and their relations to
the discriminant, and, with Cayley and Sylvester, studied those
differential equations which are satisfied by invariants and
covariants of binary quantics. His symbolic method has been carried
on by Clebsch, Gordan, and more recently by Study (1889) and Stroh
(1890), in lines quite different from those of the English school.

In France Hermite early took up the work (1851). He discovered
(1854) the law of reciprocity that to every covariant or invariant
of degree $\rho$ and order $r$ of a form of the $m$th order
corresponds also a covariant or invariant of degree $m$ and of order
$r$ of a form of the $\rho$th order. At the same time (1854)
Brioschi joined the movement, and his contributions have been among
the most valuable. Salmon's Higher Plane Curves (1852) and Higher
Algebra (1859) should also be mentioned as marking an epoch in the
theory.

Gordan entered the field, as a critic of Cayley, in 1868. He added
greatly to the theory, especially by his theorem on the Endlichkeit
des Formensystems, the proof for which has since been
simplified. This theory of the finiteness of the number of
invariants and covariants of a binary form has since been extended
by Peano (1882), Hilbert (1884), and Mertens (1886). Hilbert (1890)
succeeded in showing the finiteness of the complete systems for
forms in $n$ variables, a proof which Story has simplified.

Clebsch\footnote{Klein's Evanston Lectures, Lect. I.} did more than
any other to introduce into Germany the work of Cayley and
Sylvester, interpreting the projective geometry by their theory of
invariants, and correlating it with Riemann's theory of
functions. Especially since the publication of his work on forms
(1871) the subject has attracted such scholars as Weierstrass,
Kronecker, Mansion, Noether, Hilbert, Klein, Lie, Beltrami,
Burkhardt, and many others. On binary forms Fa\`a di Bruno's work is
well known, as is Study's (1889) on ternary forms. De Toledo (1889)
and Elliott (1895) have published treatises on the subject.

Dublin University has also furnished a considerable corps of
contributors, among whom MacCullagh, Hamilton, Salmon, Michael and
Ralph Roberts, and Burnside may be especially mentioned. Burnside,
who wrote the latter part of Burnside and Panton's Theory of
Equations, has set forth a method of transformation which is fertile
in geometric interpretation and binds together binary and certain
ternary forms.

The equivalence problem of quadratic and bilinear forms has
attracted the attention of Weierstrass, Kronecker, Christoffel,
Frobenius, Lie, and more recently of Rosenow (Crelle, 108), Werner
(1889), Killing (1890), and Scheffers (1891). The equivalence
problem of non-quadratic forms has been studied by
Christoffel. Schwarz (1872), Fuchs (1875-76), Klein (1877, 1884),
Brioschi (1877), and Maschke (1887) have contributed to the theory
of forms with linear transformations into themselves. Cayley
(especially from 1870) and Sylvester (1877) have worked out the
methods of denumeration by means of generating
functions. Differential invariants have been studied by Sylvester,
MacMahon, and Hammond. Starting from the differential invariant,
which Cayley has termed the Schwarzian derivative, Sylvester (1885)
has founded the theory of reciprocants, to which MacMahon, Hammond,
Leudesdorf, Elliott, Forsyth, and Halphen have
contributed. Canonical forms have been studied by Sylvester (1851),
Cayley, and Hermite (to whom the term ``canonical form'' is due),
and more recently by Rosanes (1873), Brill (1882), Gundelfinger
(1883), and Hilbert (1886).

The Geometric Theory of Binary Forms may be traced to Poncelet and
his followers. But the modern treatment has its origin in connection
with the theory of elliptic modular functions, and dates from
Dedekind's letter to Borchardt (Crelle, 1877). The names of Klein
and Hurwitz are prominent in this connection. On the method of nets
(r\'eseaux), another geometric treatment of binary quadratic forms
Gauss (1831), Dirichlet (1850), and Poincar\'e (1880) have written.

\chapter{CALCULUS.}

The Differential and Integral Calculus,\footnote{Williamson, B.,
Infinitesimal Calculus, Encyclop\ae{}dia Britannica, 9th edition;
Cantor, M., Geschichte der Mathematik, Vol. III, pp. 150-316;
Vivanti, G., Note sur l'histoire de l'infiniment petit, Bibliotheca
Mathematica, 1894, p. 1; Mansion, P., Esquisse de l'histoire du calcul
infinit\'esimal, Ghent, 1887. Le deux centi\`eme anniversaire
de l'invention du calcul diff\'erentiel; Mathesis, Vol. IV, p. 163.}
dating from Newton and Leibniz, was quite complete in its general
range at the close of the eighteenth century. Aside from the study
of first principles, to which Gauss, Cauchy, Jordan, Picard, M\'eray,
and those whose names are mentioned in connection with the theory of
functions, have contributed, there must be mentioned the development
of symbolic methods, the theory of definite integrals, the calculus
of variations, the theory of differential equations, and the
numerous applications of the Newtonian calculus to physical
problems. Among those who have prepared noteworthy general treatises
are Cauchy (1821), Raabe (1839-47), Duhamel (1856), Sturm (1857-59),
Bertrand (1864), Serret (1868), Jordan (2d ed., 1893), and Picard
(1891-93). A recent contribution to analysis which promises to be
valuable is Oltramare's Calcul de G\'en\'eralization (1893).

Abel seems to have been the first to consider in a general way the
question as to what differential expressions can be integrated in a
finite form by the aid of ordinary functions, an investigation
extended by Liouville. Cauchy early undertook the general theory of
determining definite integrals, and the subject has been prominent
during the century. Frullani's theorem (1821), Bierens de Haan's
work on the theory (1862) and his elaborate tables (1867),
Dirichlet's lectures (1858) embodied in Meyer's treatise (1871), and
numerous memoirs of Legendre, Poisson, Plana, Raabe, Sohncke,
Schl\"omilch, Elliott, Leudesdorf, and Kronecker are among the
noteworthy contributions.

Eulerian Integrals were first studied by Euler and afterwards
investigated by Legendre, by whom they were classed as Eulerian
integrals of the first and second species, as follows: $\int_0^1
x^{n-1}(1 - x)^{n-1}dx$, $\int_0^\infty e^{-x} x^{n-1}dx$, although
these were not the exact forms of Euler's study. If $n$ is
integral, it follows that $\int_0^\infty e^{-x}x^{n-1}dx = n!$, but
if $n$ is fractional it is a transcendent function. To it
Legendre assigned the symbol $\Gamma$, and it is now called the
gamma function. To the subject Dirichlet has contributed an
important theorem (Liouville, 1839), which has been elaborated by
Liouville, Catalan, Leslie Ellis, and others. On the evaluation of
$\Gamma x$ and $\log \Gamma x$ Raabe (1843-44), Bauer (1859), and
Gudermann (1845) have written. Legendre's great table appeared in
1816.

Symbolic Methods may be traced back to Taylor, and the analogy
between successive differentiation and ordinary exponentials had
been observed by numerous writers before the nineteenth
century. Arbogast (1800) was the first, however, to separate the
symbol of operation from that of quantity in a differential
equation. Fran\c{c}ois (1812) and Servois (1814) seem to have been
the first to give correct rules on the subject. Hargreave (1848)
applied these methods in his memoir on differential equations, and
Boole freely employed them. Grassmann and Hankel made great use of
the theory, the former in studying equations, the latter in his
theory of complex numbers.

The Calculus of Variations\footnote{Carll, L. B., Calculus of
Variations, New York, 1885, Chap. V; Todhunter, I., History of the
Progress of the Calculus of Variations, London, 1861; Reiff, R., Die
Anf\"ange der Variationsrechnung,
Mathematisch-naturwissenschaftliche Mittheilungen, T\"ubingen,
1887, p. 90.} may be said to begin with a problem of Johann
Bernoulli's (1696). It immediately occupied the attention of Jakob
Bernoulli and the Marquis de l'H\^opital, but Euler first elaborated
the subject. His contributions began in 1733, and his Elementa
Calculi Variationum gave to the science its name. Lagrange
contributed extensively to the theory, and Legendre (1786) laid down
a method, not entirely satisfactory, for the discrimination of
maxima and minima. To this discrimination Brunacci (1810), Gauss
(1829), Poisson (1831), Ostrogradsky (1834), and Jacobi (1837) have
been among the contributors. An important general work is that of
Sarrus (1842) which was condensed and improved by Cauchy
(1844). Other valuable treatises and memoirs have been written by
Strauch (1849), Jellett (1850), Hesse (1857), Clebsch (1858), and
Carll (1885), but perhaps the most important work of the century is
that of Weierstrass. His celebrated course on the theory is
epoch-making, and it may be asserted that he was the first to place
it on a firm and unquestionable foundation.

The Application of the Infinitesimal Calculus to problems in physics
and astronomy was contemporary with the origin of the science. All
through the eighteenth century these applications were multiplied,
until at its close Laplace and Lagrange had brought the whole range
of the study of forces into the realm of analysis. To Lagrange
(1773) we owe the introduction of the theory of the
potential\footnote{Bacharach, M., Abriss der Geschichte der
Potentialtheorie, 1883. This contains an extensive bibliography.}
into dynamics, although the name ``potential function'' and the
fundamental memoir of the subject are due to Green (1827, printed in
1828). The name ``potential'' is due to Gauss (1840), and the
distinction between potential and potential function to
Clausius. With its development are connected the names of Dirichlet,
Riemann, Neumann, Heine, Kronecker, Lipschitz, Christoffel,
Kirchhoff, Beltrami, and many of the leading physicists of the
century.

It is impossible in this place to enter into the great variety of
other applications of analysis to physical problems. Among them are
the investigations of Euler on vibrating chords; Sophie Germain on
elastic membranes; Poisson, Lam\'e, Saint-Venant, and Clebsch on
the elasticity of three-dimensional bodies; Fourier on heat
diffusion; Fresnel on light; Maxwell, Helmholtz, and Hertz on
electricity; Hansen, Hill, and Gyld\'en on astronomy; Maxwell on
spherical harmonics; Lord Rayleigh on acoustics; and the
contributions of Dirichlet, Weber, Kirchhoff, F. Neumann, Lord
Kelvin, Clausius, Bjerknes, MacCullagh, and Fuhrmann to physics in
general. The labors of Helmholtz should be especially mentioned,
since he contributed to the theories of dynamics, electricity, etc.,
and brought his great analytical powers to bear on the fundamental
axioms of mechanics as well as on those of pure mathematics.

\chapter{DIFFERENTIAL EQUATIONS.}

The Theory of Differential Equations\footnote{Cantor, M.,
Geschichte der Mathematik, Vol. III, p. 429; Schlesinger, L.,
Handbuch der
Theorie der linearen Differentialgleichungen, Vol. I, 1895, an
excellent historical view; review by Mathews in Nature, Vol. LII,
p. 313; Lie, S., Zur allgemeinen Theorie der partiellen
Differentialgleichungen, Berichte \"uber die Verhandlungen der
Gesellschaft der Wissenschaften zu Leipzig, 1895; Mansion, P.,
Theorie der partiellen Differentialgleichungen ter Ordnung, German
by Maser, Leipzig, 1892, excellent on history; Craig, T., Some of
the Developments in the Theory of Ordinary Differential Equations,
1878-1893, Bulletin New York Mathematical Society, Vol. II, p. 119 ;
Goursat, E., Le\c{c}ons sur l'int\'egration des \'equations aux
d\'eriv\'ees partielles du premier ordre, Paris, 1895; Burkhardt,
H., and Heffier, L., in Mathematical Papers of Chicago Congress,
p.13 and p. 96.} has been called by Lie\footnote{``In der ganzen
modernen Mathematik ist die Theorie der Differentialgleichungen die
wichtigste Disciplin.''} the most important of modern
mathematics. The influence of geometry, physics, and astronomy,
starting with Newton and Leibniz, and further manifested through the
Bernoullis, Riccati, and Clairaut, but chiefly through d'Alembert
and Euler, has been very marked, and especially on the theory of
linear partial differential equations with constant coefficients.
The first method of integrating linear ordinary differential
equations with constant coefficients is due to Euler, who made the
solution of his type, $\frac {d^{n}y} {dx^{n}} + A_{1}\frac
{d^{n-1}y} {dx^{n-1}} + \cdots + A_{n}y = 0$, depend on that of the
algebraic equation of the
$n$th degree, $F(z) = z^{n} + A_{1}z^{n-1} + \cdots + An = 0$, in
which $z^{k}$ takes the place of $\frac {d^{k}y} {dx^{k}} (k = 1, 2,
\cdots, n)$. This equation $F(z) = 0$, is the ``characteristic''
equation considered later by Monge and Cauchy.

The theory of linear partial differential equations may be said to
begin with Lagrange (1779 to 1785). Monge (1809) treated ordinary
and partial differential equations of the first and second order,
uniting the theory to geometry, and introducing the notion of the
``characteristic,'' the curve represented by $F(z) = 0$, which has
recently been investigated by Darboux, Levy, and Lie. Pfaff (1814,
1815) gave the first general method of integrating partial
differential equations of the first order, a method of which Gauss
(1815) at once recognized the value and of which he gave an
analysis. Soon after, Cauchy (1819) gave a simpler method, attacking
the subject from the analytical standpoint, but using the Monge
characteristic. To him is also due the theorem, corresponding to the
fundamental theorem of algebra, that every differential equation
defines a function expressible by means of a convergent series, a
proposition more simply proved by Briot and Bouquet, and also by
Picard (1891). Jacobi (1827) also gave an analysis of Pfaff's
method, besides developing an original one (1836) which Clebsch
published (1862). Clebsch's own method appeared in 1866, and others
are due to Boole (1859), Korkine (1869), and A. Mayer
(1872). Pfaff's problem has been a prominent subject of
investigation, and with it are connected the names of Natani (1859),
Clebsch (1861, 1862), DuBois-Reymond (1869), Cayley, Baltzer,
Frobenius, Morera, Darboux, and Lie. The next great improvement in
the theory of partial differential equations of the first order is
due to Lie (1872), by whom the whole subject has been placed on a
rigid foundation. Since about 1870, Darboux, Kovalevsky, M\'eray,
Mansion, Graindorge, and Imschenetsky have been prominent in this
line. The theory of partial differential equations of the second
and higher orders, beginning with Laplace and Monge, was notably
advanced by Amp\`ere (1840). Imschenetsky\footnote{Grunert's Archiv
f\"ur Mathematik, Vol. LIV.} has summarized the contributions to
1873, but the theory remains in an imperfect state.

The integration of partial differential equations with three or more
variables was the object of elaborate investigations by Lagrange,
and his name is still connected with certain subsidiary
equations. To him and to Charpit, who did much to develop the
theory, is due one of the methods for integrating the general
equation with two variables, a method which now bears Charpit's name.

The theory of singular solutions of ordinary and partial
differential equations has been a subject of research from the time
of Leibniz, but only since the middle of the present century has it
received especial attention. A valuable but little-known work on the
subject is that of Houtain (1854). Darboux (from 1873) has been a
leader in the theory, and in the geometric interpretation of these
solutions he has opened a field which has been worked by various
writers, notably Casorati and Cayley. To the latter is due (1872)
the theory of singular solutions of differential equations of the
first order as at present accepted.

The primitive attempt in dealing with differential equations had in


view a reduction to quadratures. As it had been the hope of
eighteenth-century algebraists to find a method for solving the
general equation of the $n$th degree, so it was the hope of analysts
to find a general method for integrating any differential
equation. Gauss (1799) showed, however, that the differential
equation meets its limitations very soon unless complex numbers are
introduced. Hence analysts began to substitute the study of
functions, thus opening a new and fertile field. Cauchy was the
first to appreciate the importance of this view, and the modern
theory may be said to begin with him. Thereafter the real question
was to be, not whether a solution is possible by means of known
functions or their integrals, but whether a given differential
equation suffices for the definition of a function of the
independent variable or variables, and if so, what are the
characteristic properties of this function.

Within a half-century the theory of ordinary differential equations
has come to be one of the most important branches of analysis, the
theory of partial differential equations remaining as one still to
be perfected. The difficulties of the general problem of integration
are so manifest that all classes of investigators have confined
themselves to the properties of the integrals in the neighborhood of
certain given points. The new departure took its greatest
inspiration from two memoirs by Fuchs (Crelle, 1866, 1868), a work
elaborated by Thom\'e and Frobenius. Collet has been a prominent
contributor since 1869, although his method for integrating a
non-linear system was communicated to Bertrand in 1868.
Clebsch\footnote{Klein's Evanston Lectures, Lect. I.} (1873) attacked
the theory along lines parallel to those followed in his theory of
Abelian integrals. As the latter can be classified according to the
properties of the fundamental curve which remains unchanged under a
rational transformation, so Clebsch proposed to classify the
transcendent functions defined by the differential equations
according to the invariant properties of the corresponding surfaces
$f = 0$ under rational one-to-one transformations.

Since 1870 Lie's\footnote{Klein's Evanston Lectures, Lect. II,
III.} labors have put the entire theory of differential equations
on a more satisfactory foundation. He has shown that the integration
theories of the older mathematicians, which had been looked upon as
isolated, can by the introduction of the concept of continuous
groups of transformations be referred to a common source, and that
ordinary differential equations which admit the same infinitesimal
transformations present like difficulties of integration. He has
also emphasized the subject of transformations of contact
(Ber\"uhrungstransformationen) which underlies so much of the recent
theory. The modern school has also turned its attention to the
theory of differential invariants, one of fundamental importance and
one which Lie has made prominent. With this theory are associated
the names of Cayley, Cockle, Sylvester, Forsyth, Laguerre, and
Halphen. Recent writers have shown the same tendency noticeable in
the work of Monge and Cauchy, the tendency to separate into two
schools, the one inclining to use the geometric diagram, and
represented by Schwarz, Klein, and Goursat, the other adhering to
pure analysis, of which Weierstrass, Fuchs, and Frobenius are
types. The work of Fuchs and the theory of elementary divisors have
formed the basis of a late work by Sauvage (1895). Poincar\'e's
recent contributions are also very notable. His theory of Fuchsian
equations (also investigated by Klein) is connected with the general
theory. He has also brought the whole subject into close relations
with the theory of functions. Appell has recently contributed to the
theory of linear differential equations transformable into
themselves by change of the function and the variable. Helge von
Koch has written on infinite determinants and linear differential
equations. Picard has undertaken the generalization of the work of
Fuchs and Poincar\'e in the case of differential equations of the
second order. Fabry (1885) has generalized the normal integrals of
Thom\'e, integrals which Poincar\'e has called ``int\'egrales
anormales,'' and which Picard has recently studied. Riquier has
treated the question of the existence of integrals in any
differential system and given a brief summary of the history to
1895.\footnote{Riquier, C., M\'emoire sur l'existence des
int\'egrales dans un syst\`eme differentiel quelconque,
etc. M\'emoires des Savants \'etrangers, Vol. XXXII, No. 3.} The
number of contributors in recent times is very great, and includes,
besides those already mentioned, the names of Brioschi,
K\"onigsberger, Peano, Graf, Hamburger, Graindorge, Schl\"afli,
Glaisher, Lommel, Gilbert, Fabry, Craig, and Autonne.

\chapter{INFINITE SERIES.}

The Theory of Infinite Series\footnote{Cantor, M., Geschichte der
Mathematik, Vol. III, pp. 53, 71; Reiff, R., Geschichte der
unendlichen Reihen, T\"ubingen, 1889; Cajori, F., Bulletin New York
Mathematical Society, Vol. I, p. 184; History of Teaching of
Mathematics in United States, p. 361.} in its historical
development has been divided by Reiff into three periods: (1) the
period of Newton and Leibniz, that of its introduction; (2) that of
Euler, the formal period; (3) the modern, that of the scientific
investigation of the validity of infinite series, a period beginning
with Gauss. This critical period begins with the publication of
Gauss's celebrated memoir on the series $1 +
\frac{\alpha.\beta}{1.\gamma}x +
\frac{\alpha.(\alpha+1).\beta.(\beta+1)}{1.2.\gamma.(\gamma+1)}x^2 +
\cdots$, in 1812. Euler had already considered this series, but Gauss
was the first to master it, and under the name ``hypergeometric
series'' (due to Pfaff) it has since occupied the attention of
Jacobi, Kummer, Schwarz, Cayley, Goursat, and numerous others. The
particular series is not so important as is the standard of
criticism which Gauss set up, embodying the simpler criteria of
convergence and the questions of remainders and the range of
convergence.

Gauss's contributions were not at once appreciated, and the next to
call attention to the subject was Cauchy (1821), who may be
considered the founder of the theory of convergence and divergence
of series. He was one of the first to insist on strict tests of
convergence; he showed that if two series are convergent their
product is not necessarily so; and with him begins the discovery of
effective criteria of convergence and divergence. It should be
mentioned, however, that these terms had been introduced long before
by Gregory (1668), that Euler and Gauss had given various criteria,
and that Maclaurin had anticipated a few of Cauchy's discoveries.
Cauchy advanced the theory of power series by his expansion of a
complex function in such a form. His test for convergence is still
one of the most satisfactory when the integration involved is
possible.

Abel was the next important contributor. In his memoir (1826) on the
series $1 + \frac{m}{1}x + \frac{m(m-1)}{2!}x^2 + \cdots$ he
corrected certain of Cauchy's conclusions, and gave a completely
scientific summation of the series for complex values of $m$ and $x$.
He was emphatic against the reckless use of series, and showed the
necessity of considering the subject of continuity in questions of
convergence.

Cauchy's methods led to special rather than general criteria, and
the same may be said of Raabe (1832), who made the first elaborate
investigation of the subject, of De Morgan (from 1842), whose
logarithmic test DuBois-Reymond (1873) and Pringsheim (1889) have
shown to fail within a certain region; of Bertrand (1842), Bonnet
(1843), Malmsten (1846, 1847, the latter without integration);
Stokes (1847), Paucker (1852), Tch\'ebichef (1852), and Arndt
(1853). General criteria began with Kummer (1835), and have been
studied by Eisenstein (1847), Weierstrass in his various
contributions to the theory of functions, Dini (1867),
DuBois-Reymond (1873), and many others. Pringsheim's (from 1889)
memoirs present the most complete general theory.

The Theory of Uniform Convergence was treated by Cauchy (1821), his
limitations being pointed out by Abel, but the first to attack it
successfully were Stokes and Seidel (1847-48). Cauchy took up the
problem again (1853), acknowledging Abel's criticism, and reaching
the same conclusions which Stokes had already found. Thom\'e used the
doctrine (1866), but there was great delay in recognizing the
importance of distinguishing between uniform and non-uniform
convergence, in spite of the demands of the theory of functions.

Semi-Convergent Series were studied by Poisson (1823), who also gave
a general form for the remainder of the Maclaurin formula. The most
important solution of the problem is due, however, to Jacobi (1834),
who attacked the question of the remainder from a different
standpoint and reached a different formula. This expression was
also worked out, and another one given, by Malmsten (1847).
Schl\"omilch (Zeitschrift, Vol.I, p. 192, 1856) also
improved Jacobi's remainder, and showed the relation between the
remainder and Bernoulli's function $F(x) = 1^n + 2^n + \cdots + (x -
1)^n$. Genocchi (1852) has further contributed to the theory.

Among the early writers was Wronski, whose ``loi supr\^eme'' (1815)
was hardly recognized until Cayley (1873) brought it into
prominence. Transon (1874), Ch. Lagrange (1884), Echols, and
Dickstein\footnote{Bibliotheca Mathematica, 1892-94; historical.}
have published of late various memoirs on the subject.

Interpolation Formulas have been given by various writers from
Newton to the present time. Lagrange's theorem is well known,
although Euler had already given an analogous form, as are also
Olivier's formula (1827), and those of Minding (1830), Cauchy
(1837), Jacobi (1845), Grunert (1850, 1853), Christoffel (1858), and
Mehler (1864).

Fourier's Series\footnote{Historical Summary by B\^ocher, Chap. IX
of Byerly's Fourier's Series and Spherical Harmonics, Boston, 1893;
Sachse, A., Essai historique sur la repr\'esentation d'une fonction
\ldots par une s\'erie trigonom\'etrique. Bulletin des Sciences
math\'ematiques, Part I, 1880, pp. 43, 83.} were being investigated
as the result of physical considerations at the same time that
Gauss, Abel, and Cauchy were working out the theory of infinite
series. Series for the expansion of sines and cosines, of multiple
arcs in powers of the sine and cosine of the arc had been treated by
Jakob Bernoulli (1702) and his brother Johann (1701) and still
earlier by Vi\`ete. Euler and Lagrange had simplified the subject,
as have, more recently, Poinsot, Schr\"oter, Glaisher, and
Kummer. Fourier (1807) set for himself a different problem, to
expand a given function of $x$ in terms of the sines or cosines of
multiples of $x$, a problem which he embodied in his Th\'eorie
analytique de la Chaleur (1822). Euler had already given the
formulas for determining the coefficients in the series; and
Lagrange had passed over them without recognizing their value, but
Fourier was the first to assert and attempt to prove the general
theorem. Poisson (1820-23) also attacked the problem from a
different standpoint. Fourier did not, however, settle the question
of convergence of his series, a matter left for Cauchy (1826) to
attempt and for Dirichlet (1829) to handle in a thoroughly
scientific manner. Dirichlet's treatment (Crelle, 1829), while
bringing the theory of trigonometric series to a temporary
conclusion, has been the subject of criticism and improvement by
Riemann (1854), Heine, Lipschitz, Schl\"afli, and
DuBois-Reymond. Among other prominent contributors to the theory of
trigonometric and Fourier series have been Dini, Hermite, Halphen,
Krause, Byerly and Appell.

\chapter{THEORY OF FUNCTIONS.}

The Theory of Functions\footnote{Brill, A., and Noether, M., Die
Entwickelung der Theorie der algebraischen Functionen in alterer
und neuerer Zeit, Bericht erstattet der Deutschen
Mathematiker-Vereinigung, Jahresbericht, Vol. II, pp. 107-566,
Berlin, 1894; K\"onigsberger, L., Zur Geschichte der Theorie der
elliptischen Transcendenten in den Jahren 1826-29, Leipzig, 1879;
Williamson, B., Infinitesimal Calculus, Encyclop\ae{}dia Britannica;
Schlesinger, L., Differentialgleichungen, Vol. I, 1895; Casorati,
F., Teorica delle funzioni di variabili complesse, Vol. I, 1868;
Klein's Evanston Lectures. For bibliography and historical notes,
see Harkness and Morley's Theory of Functions, 1893, and Forsyth's
Theory of Functions, 1893; Enestr\"om, G., Note historique sur les
symboles \ldots Bibliotheca Mathematica, 1891, p. 89.} may be said to
have its first development in Newton's works, although algebraists
had already become familiar with irrational functions in considering
cubic and quartic equations. Newton seems first to have grasped the
idea of such expressions in his consideration of symmetric functions
of the roots of an equation. The word was employed by Leibniz
(1694), but in connection with the Cartesian geometry. In its modern
sense it seems to have been first used by Johann Bernoulli, who
distinguished between algebraic and transcendent functions. He also
used (1718) the function symbol $\phi$. Clairaut (1734) used $\Pi
x$, $\Phi x$, $\Delta x$, for various functions of $x$, a symbolism
substantially followed by d'Alembert (1747) and Euler
(1753). Lagrange (1772, 1797, 1806) laid the foundations for the
general theory, giving to the symbol a broader meaning, and to the
symbols $f$, $\phi$, $F$, $\cdots$, $f^{\prime}$, $\phi^{\prime}$,
$F^{\prime}$, $\cdots$ their modern signification. Gauss contributed
to the theory, especially in his proofs of the fundamental theorem
of algebra, and discussed and gave name to the theory of ``conforme
Abbildung,'' the ``orthomorphosis'' of Cayley.

Making Lagrange's work a point of departure, Cauchy so greatly
developed the theory that he is justly considered one of its
founders. His memoirs extend over the period 1814-1851, and cover
subjects like those of integrals with imaginary limits, infinite
series and questions of convergence, the application of the
infinitesimal calculus to the theory of complex numbers, the
investigation of the fundamental laws of mathematics, and numerous
other lines which appear in the general theory of functions as
considered to-day. Originally opposed to the movement started by
Gauss, the free use of complex numbers, he finally became, like
Abel, its advocate. To him is largely due the present orientation of
mathematical research, making prominent the theory of functions,
distinguishing between classes of functions, and placing the whole
subject upon a rigid foundation. The historical development of the
general theory now becomes so interwoven with that of special
classes of functions, and notably the elliptic and Abelian, that
economy of space requires their treatment together, and hence a
digression at this point.

The Theory of Elliptic Functions\footnote{Enneper, A., Elliptische
Funktionen, Theorie und Geschichte, Halle, 1890; K\"onigsberger, L.,
Zur Geschichte der Theorie der elliptischen Transcendenten in den
Jahren 1826-29, Leipzig, 1879.} is usually referred for its origin
to Landen's (1775) substitution of two elliptic arcs for a single
hyperbolic arc. But Jakob Bernoulli (1691) had suggested the idea of
comparing non-congruent arcs of the same curve, and Johann had
followed up the investigation. Fagnano (1716) had made similar
studies, and both Maclaurin (1742) and d'Alembert (1746) had come
upon the borderland of elliptic functions. Euler (from 1761) had
summarized and extended the rudimentary theory, showing the
necessity for a convenient notation for elliptic arcs, and
prophesying (1766) that ``such signs will afford a new sort of
calculus of which I have here attempted the exposition of the first
elements.'' Euler's investigations continued until about the time of
his death (1783), and to him Legendre attributes the foundation of
the theory. Euler was probably never aware of Landen's discovery.

It is to Legendre, however, that the theory of elliptic functions is
largely due, and on it his fame to a considerable degree
depends. His earlier treatment (1786) almost entirely substitutes a
strict analytic for the geometric method. For forty years he had the
theory in hand, his labor culminating in his Trait\'e des Fonctions
elliptiques et des Int\'egrales Eul\'eriennes (1825-28). A surprise
now awaiting him is best told in his own words: ``Hardly had my work
seen the light--its name could scarcely have become known to
scientific foreigners,--when I learned with equal surprise and
satisfaction that two young mathematicians, MM. Jacobi of
K\"onigsberg and Abel of Christiania, had succeeded by their own
studies in perfecting considerably the theory of elliptic functions
in its highest parts.'' Abel began his contributions to the theory
in 1825, and even then was in possession of his fundamental theorem
which he communicated to the Paris Academy in 1826. This
communication being so poorly transcribed was not published in full
until 1841, although the theorem was sent to Crelle (1829) just
before Abel's early death. Abel discovered the double periodicity of
elliptic functions, and with him began the treatment of the elliptic
integral as a function of the amplitude.

Jacobi, as also Legendre and Gauss, was especially cordial in praise
of the delayed theorem of the youthful Abel. He calls it a
``monumentum \ae{}re perennius,'' and his name ``das Abel'sche
Theorem'' has since attached to it. The functions of multiple
periodicity to which it refers have been called Abelian
Functions. Abel's work was early proved and elucidated by Liouville
and Hermite. Serret and Chasles in the Comptes Rendus, Weierstrass
(1853), Clebsch and Gordan in their Theorie der Abel'schen
Functionen (1866), and Briot and Bouquet in their two treatises have
greatly elaborated the theory. Riemann's\footnote{Klein, Evanston
Lectures, p. 3; Riemann and Modern Mathematics, translated by
Ziwet, Bulletin of American Mathematical Society, Vol. I, p. 165;
Burkhardt, H., Vortrag uber Riemann, G\"ottingen, 1892.} (1857)
celebrated memoir in Crelle presented the subject in such a novel
form that his treatment was slow of acceptance. He based the theory
of Abelian integrals and their inverse, the Abelian functions, on
the idea of the surface now so well known by his name, and on the
corresponding fundamental existence theorems. Clebsch, starting from
an algebraic curve defined by its equation, made the subject more
accessible, and generalized the theory of Abelian integrals to a
theory of algebraic functions with several variables, thus creating
a branch which has been developed by Noether, Picard, and
Poincar\'e. The introduction of the theory of invariants and
projective geometry into the domain of hyperelliptic and Abelian
functions is an extension of Clebsch's scheme. In this extension, as
in the general theory of Abelian functions, Klein has been a
leader. With the development of the theory of Abelian functions is
connected a long list of names, including those of Schottky,
Humbert, C. Neumann, Fricke, K\"onigsberger, Prym, Schwarz,
Painlev\'e, Hurwitz, Brioschi, Borchardt, Cayley, Forsyth, and
Rosenhain, besides others already mentioned.

Returning to the theory of elliptic functions, Jacobi (1827) began
by adding greatly to Legendre's work. He created a new notation and
gave name to the ``modular equations'' of which he made use. Among
those who have written treatises upon the elliptic-function theory
are Briot and Bouquet, Laurent, Halphen, K\"onigsberger, Hermite,
Dur\`ege, and Cayley, The introduction of the subject into the
Cambridge Tripos (1873), and the fact that Cayley's only book was
devoted to it, have tended to popularize the theory in England.

The Theory of Theta Functions was the simultaneous and independent
creation of Jacobi and Abel (1828). Gauss's notes show that he was
aware of the properties of the theta functions twenty years earlier,
but he never published his investigations. Among the leading
contributors to the theory are Rosenhain (1846, published in 1851)
and G\"opel (1847), who connected the double theta functions with
the theory of Abelian functions of two variables and established the
theory of hyperelliptic functions in a manner corresponding to the
Jacobian theory of elliptic functions. Weierstrass has also
developed the theory of theta functions independently of the form of
their space boundaries, researches elaborated by K\"onigsberger
(1865) to give the addition theorem. Riemann has completed the
investigation of the relation between the theory of the theta and
the Abelian functions, and has raised theta functions to their
present position by making them an essential part of his theory of
Abelian integrals. H.~J.~S.~Smith has included among his
contributions to this subject the theory of omega functions. Among
the recent contributors are Krazer and Prym (1892), and Wirtinger
(1895).

Cayley was a prominent contributor to the theory of periodic
functions. His memoir (1845) on doubly periodic functions extended
Abel's investigations on doubly infinite products. Euler had given
singly infinite products for $\sin x$, $\cos x$, and Abel had
generalized these, obtaining for the elementary doubly periodic
functions expressions for $\hbox{sn} x$, $\hbox{cn} x$, $\hbox{dn}
x$. Starting from these expressions of Abel's Cayley laid a complete
foundation for his theory of elliptic functions. Eisenstein (1847)
followed, giving a discussion from the standpoint of pure analysis,
of a general doubly infinite product, and his labors, as
supplemented by Weierstrass, are classic.

The General Theory of Functions has received its present form
largely from the works of Cauchy, Riemann, and
Weierstrass. Endeavoring to subject all natural laws to
interpretation by mathematical formulas, Riemann borrowed his
methods from the theory of the potential, and found his inspiration
in the contemplation of mathematics from the standpoint of the
concrete. Weierstrass, on the other hand, proceeded from the purely
analytic point of view. To Riemann\footnote{Klein, F., Riemann and
Modern Mathematics, translated by Ziwet, Bulletin of American
Mathematical Society, Vol. I, p. 165.} is due the idea of making
certain partial differential equations, which express the
fundamental properties of all functions, the foundation of a general
analytical theory, and of seeking criteria for the determination of
an analytic function by its discontinuities and boundary
conditions. His theory has been elaborated by Klein (1882, and
frequent memoirs) who has materially extended the theory of
Riemann's surfaces. Clebsch, L\"uroth, and later writers have based on
this theory their researches on loops. Riemann's speculations were
not without weak points, and these have been fortified in connection
with the theory of the potential by C. Neumann, and from the
analytic standpoint by Schwarz.

In both the theory of general and of elliptic and other functions,
Clebsch was prominent. He introduced the systematic consideration
of algebraic curves of deficiency 1, bringing to bear on the theory
of elliptic functions the ideas of modern projective geometry. This
theory Klein has generalized in his Theorie der elliptischen
Modulfunctionen, and has extended the method to the theory of
hyperelliptic and Abelian functions.

Following Riemann came the equally fundamental and original and more
rigorously worked out theory of Weierstrass. His early lectures on
functions are justly considered a landmark in modern mathematical
development. In particular, his researches on Abelian transcendents
are perhaps the most important since those of Abel and Jacobi. His
contributions to the theory of elliptic functions, including the
introduction of the function $\wp(u)$, are also of great
importance. His contributions to the general function theory
include much of the symbolism and nomenclature, and many
theorems. He first announced (1866) the existence of natural limits
for analytic functions, a subject further investigated by Schwarz,
Klein, and Fricke. He developed the theory of functions of complex
variables from its foundations, and his contributions to the theory
of functions of real variables were no less marked.

Fuchs has been a prominent contributor, in particular (1872) on the
general form of a function with essential singularities. On
functions with an infinite number of essential singularities
Mittag-Leffler (from 1882) has written and contributed a fundamental
theorem. On the classification of singularities of functions
Guichard (1883) has summarized and extended the researches, and
Mittag-Leffler and G. Cantor have contributed to the same
result. Laguerre (from 1882) was the first to discuss the ``class''
of transcendent functions, a subject to which Poincar\'e, Cesaro,
Vivanti, and Hermite have also contributed. Automorphic functions,
as named by Klein, have been investigated chiefly by Poincar\'e, who
has established their general classification. The contributors to
the theory include Schwarz, Fuchs, Cayley, Weber, Schlesinger, and
Burnside.

The Theory of Elliptic Modular Functions, proceeding from
Eisenstein's memoir (1847) and the lectures of Weierstrass on
elliptic functions, has of late assumed prominence through the
influence of the Klein school. Schl\"afli (1870), and later Klein,
Dyck, Gierster, and Hurwitz, have worked out the theory which Klein
and Fricke have embodied in the recent Vorlesungen
\"uber die Theorie der elliptischen Modulfunctionen
(1890-92). In this theory the memoirs of Dedekind (1877), Klein
(1878), and Poincar\'e (from 1881) have been among the most
prominent.

For the names of the leading contributors to the general and special
theories, including among others Jordan, Hermite, H\"older, Picard,
Biermann, Darboux, Pellet, Reichardt, Burkhardt, Krause, and
Humbert, reference must be had to the Brill-Noether Bericht.

Of the various special algebraic functions space allows mention of
but one class, that bearing Bessel's name. Bessel's
functions\footnote{B\^ocher, M., A bit of mathematical history,
Bulletin of New York Mathematical Society, Vol. II, p. 107.} of
the zero order arefound in memoirs of Daniel Bernoulli (1732) and
Euler (1764), and before the end of the eighteenth century all the
Bessel functions of the first kind and integral order had been used.
Their prominence as special functions is due, however, to
Bessel (1816-17), who put them in their present form in 1824. Lagrange's
series (1770), with Laplace's extension (1777), had been regarded as the
best method of solving Kepler's problem (to express the variable quantities
in undisturbed planetary motion in terms of the time or mean anomaly),
and to improve this method Bessel's functions were first prominently
used. Hankel (1869), Lommel (from 1868), F.~Neumann, Heine, Graf
(1893), Gray and Mathews (1895), and others have contributed to the
theory. Lord Rayleigh (1878) has shown the relation between
Bessel's and Laplace's functions, but they are nevertheless looked
upon as a distinct system of transcendents. Tables of Bessel's
functions were prepared by Bessel (1824), by Hansen (1843), and by
Meissel (1888).

\chapter{PROBABILITIES AND LEAST SQUARES.}

The Theory of Probabilities and Errors\footnote{Merriman, M., Method
of Least Squares, New York, 1884, p. 182; Transactions of
Connecticut Academy, 1877, Vol. IV, p. 151, with complete
bibliography; Todhunter, I., History of the Mathematical Theory of
Probability, 1865; Cantor, M., Geschichte der Mathematik, Vol. III,
p. 316.} is, as applied to observations, largely a
nineteenth-century development. The doctrine of probabilities dates,
however, as far back as Fermat and Pascal (1654). Huygens (1657)
gave the first scientific treatment of the subject, and Jakob
Bernoulli's Ars Conjectandi (posthumous, 1713) and De Moivre's
Doctrine of Chances (1718)\footnote{Enestr\"om, G., Review of
Cantor, Bibliotheca Mathematica, 1896, p. 20.} raised the subject
to the plane of a branch of mathematics. The theory of errors may
be traced back to Cotes's Opera Miscellanea (posthumous, 1722), but
a memoir prepared by Simpson in 1755 (printed 1756) first applied
the theory to the discussion of errors of observation. The reprint
(1757) of this memoir lays down the axioms that positive and
negative errors are equally probable, and that there are certain
assignable limits within which all errors may be supposed to fall;
continuous errors are discussed and a probability curve is given.
Laplace (1774) made the first attempt to deduce a rule for the
combination of observations from the principles of the theory of
probabilities. He represented the law of probability of errors by a
curve $y = \phi(x)$, $x$ being any error and $y$ its probability,
and laid down three properties of this curve: (1) It is symmetric as
to the $y$-axis; (2) the $x$-axis is an asymptote, the probability
of the error $\infty$ being $0$; (3) the area enclosed is $1$, it
being certain that an error exists. He deduced a formula for the
mean of three observations. He also gave (1781) a formula for the
law of facility of error (a term due to Lagrange, 1774), but one
which led to unmanageable equations. Daniel Bernoulli (1778)
introduced the principle of the maximum product of the probabilities
of a system of concurrent errors.

The Method of Least Squares is due to Legendre (1805), who
introduced it in his Nouvelles m\'ethodes pour la d\'etermination
des orbites des com\`etes. In ignorance of Legendre's contribution,
an Irish-American writer, Adrain, editor of ``The Analyst'' (1808),
first deduced the law of facility of error, $\phi(x) = ce^{-h^2
x^2}$, $c$ and $h$ being constants depending on precision of
observation. He gave two proofs, the second being essentially the
same as Herschel's (1850). Gauss gave the first proof which seems to
have been known in Europe (the third after Adrain's) in 1809. To him
is due much of the honor of placing the subject before the
mathematical world, both as to the theory and its applications.

Further proofs were given by Laplace (1810, 1812), Gauss (1823),
Ivory (1825, 1826), Hagen (1837), Bessel (1838), Donkin (1844,
1856), and Crofton (1870). Other contributors have been Ellis
(1844), De Morgan (1864), Glaisher (1872), and Schiaparelli
(1875). Peters's (1856) formula for $r$, the probable error of a
single observation, is well known.\footnote{Bulletin of New York
Mathematical Society, Vol. II, p. 57.}

Among the contributors to the general theory of probabilities in
the nineteenth century have been Laplace, Lacroix (1816), Littrow
(1833), Quetelet (1853), Dedekind (1860), Helmert (1872), Laurent
(1873), Liagre, Didion, and Pearson. De Morgan and Boole improved
the theory, but added little that was fundamentally new. Czuber has
done much both in his own contributions (1884, 1891), and in his
translation (1879) of Meyer. On the geometric side the influence of
Miller and The Educational Times has been marked, as also that of
such contributors to this journal as Crofton, McColl, Wolstenholme,
Watson, and Artemas Martin.

\chapter{ANALYTIC GEOMETRY.}

The History of Geometry\footnote{Loria, G., Il passato e il presente
delle principali teorie geometriche. Memorie Accademia Torino,
1887; translated into German by F. Schutte under the title Die
haupts\"achlichsten Theorien der Geometrie in ihrer fr\"uheren und
heutigen Entwickelung, Leipzig, 1888; Chasles, M., Aper\c{c}u
historique sur l'origine et le d\'eveloppement des m\'ethodes en
G\'eom\'etrie, 1889; Chasles, M., Rapport sur les Progr\`es de la
G\'eom\'etrie, Paris, 1870; Cayley, A., Curves, Encyclop\ae{}dia
Britannica; Klein, F., Evanston Lectures on Mathematics, New York,
1894; A. V. Braunm\"uhl, Historische Studie \"uber die organische
Erzeugung ebener Curven, Dyck's Katalog mathematischer Modelle,
1892.} may be roughly divided into the four periods: (1) The
synthetic geometry of the Greeks, practically closing with
Archimedes; (2) The birth of analytic geometry, in which the
synthetic geometry of Guldin, Desargues, Kepler, and Roberval merged
into the coordinate geometry of Descartes and Fermat; (3) 1650 to
1800, characterized by the application of the calculus to geometry,
and including the names of Newton, Leibnitz, the Bernoullis,
Clairaut, Maclaurin, Euler, and Lagrange, each an analyst rather
than a geometer; (4) The nineteenth century, the renaissance of pure
geometry, characterized by the descriptive geometry of Monge, the
modern synthetic of Poncelet, Steiner, von Staudt, and Cremona, the
modern analytic founded by Pl\"ucker, the non-Euclidean hypothesis
of Lobachevsky and Bolyai, and the more elementary geometry of the
triangle founded by Lemoine. It is quite impossible to draw the
line between the analytic and the synthetic geometry of the
nineteenth century, in their historical development, and Arts. 15
and 16 should be read together.

The Analytic Geometry which Descartes gave to the world in 1637 was
confined to plane curves, and the various important properties
common to all algebraic curves were soon discovered. To the theory
Newton contributed three celebrated theorems on the Enumeratio
linearum tertii ordinis\footnote{Ball, W.~W.~R., On Newton's
classification of cubic curves. Transactions of London Mathematical
Society, 1891, p. 104.} (1706), while others are due to Cotes
(1722), Maclaurin, and Waring (1762, 1772, etc.). The scientific
foundations of the theory of plane curves may be ascribed, however,
to Euler (1748) and Cramer (1750). Euler distinguished between
algebraic and transcendent curves, and attempted a classification of
the former. Cramer is well known for the ``paradox'' which bears his
name, an obstacle which Lam\'e (1818) finally removed from the
theory. To Cramer is also due an attempt to put the theory of
singularities of algebraic curves on a scientific foundation,
although in a modern geometric sense the theory was first treated by
Poncelet.

Meanwhile the study of surfaces was becoming prominent. Descartes
had suggested that his geometry could be extended to
three-dimensional space, Wren (1669) had discovered the two systems
of generating lines on the hyperboloid of one sheet, and Parent
(1700) had referred a surface to three coordinate planes. The
geometry of three dimensions began to assume definite shape,
however, in a memoir of Clairaut's (1731), in which, at the age of
sixteen, he solved with rare elegance many of the problems relating
to curves of double curvature. Euler (1760) laid the foundations
for the analytic theory of curvature of surfaces, attempting the
classification of those of the second degree as the ancients had
classified curves of the second order. Monge, Hachette, and other
members of that school entered into the study of surfaces with great
zeal. Monge introduced the notion of families of surfaces, and
discovered the relation between the theory of surfaces and the
integration of partial differential equations, enabling each to be
advantageously viewed from the standpoint of the other. The theory
of surfaces has attracted a long list of contributors in the
nineteenth century, including most of the geometers whose names are
mentioned in the present article.\footnote{For details see Loria,
Il passato e il presente, etc.}

M\"obius began his contributions to geometry in 1823, and four years
later published his Barycentrische Calc\"ul. In this great work he
introduced homogeneous coordinates with the attendant symmetry of
geometric formulas, the scientific exposition of the principle of
signs in geometry, and the establishment of the principle of
geometric correspondence simple and multiple. He also (1852) summed
up the classification of cubic curves, a service rendered by
Zeuthen (1874) for quartics. To the period of M\"obius also belong
Bobillier (1827), who first used trilinear coordinates, and
Bellavitis, whose contributions to analytic geometry were
extensive. Gergonne's labors are mentioned in the next article.

Of all modern contributors to analytic geometry, Pl\"ucker stands
foremost. In 1828 he published the first volume of his
Analytisch-geometrische Entwickelungen, in which appeared
the modern abridged notation, and which marks the beginning of a new
era for analytic geometry. In the second volume (1831) he sets forth
the present analytic form of the principle of duality. To him is due
(1833) the general treatment of foci for curves of higher degree,
and the complete classification of plane cubic curves (1835) which
had been so frequently tried before him. He also gave (1839) an
enumeration of plane curves of the fourth order, which Bragelogne
and Euler had attempted. In 1842 he gave his celebrated ``six
equations'' by which he showed that the characteristics of a curve
(order, class, number of double points, number of cusps, number of
double tangents, and number of inflections) are known when any three
are given. To him is also due the first scientific dual definition
of a curve, a system of tangential coordinates, and an
investigation of the question of double tangents, a question further
elaborated by Cayley (1847, 1858), Hesse (1847), Salmon (1858), and
Dersch (1874). The theory of ruled surfaces, opened by Monge, was
also extended by him. Possibly the greatest service rendered by
Pl\"ucker was the introduction of the straight line as a space
element, his first contribution (1865) being followed by his
well-known treatise on the subject (1868-69). In this work he treats
certain general properties of complexes, congruences, and ruled
surfaces, as well as special properties of linear complexes and
congruences, subjects also considered by Kummer and by Klein and
others of the modern school. It is not a little due to Pl\"ucker that
the concept of 4- and hence $n$-dimensional space, already suggested
by Lagrange and Gauss, became the subject of later
research. Riemann, Helmholtz, Lipschitz, Kronecker, Klein, Lie,
Veronese, Cayley, d'Ovidio, and many others have elaborated the
theory. The regular hypersolids in 4-dimensional space have been
the subject of special study by Scheffler, Rudel, Hoppe, Schlegel,
and Stringham.

Among Jacobi's contributions is the consideration (1836) of curves
and groups of points resulting from the intersection of algebraic
surfaces, a subject carried forward by Reye (1869). To Jacobi is
also due the conformal representation of the ellipsoid on a plane, a
treatment completed by Schering (1858). The number of examples of
conformal representation of surfaces on planes or on spheres has
been increased by Schwarz (1869) and Amstein (1872).

In 1844 Hesse, whose contributions to geometry in general are both
numerous and valuable, gave the complete theory of inflections of a
curve, and introduced the so-called Hessian curve as the first
instance of a covariant of a ternary form. He also contributed to
the theory of curves of the third order, and generalized the Pascal
and Brianchon theorems on a spherical surface. Hesse's methods have
recently been elaborated by Gundelfinger (1894).

Besides contributing extensively to synthetic geometry, Chasles
added to the theory of curves of the third and fourth degrees. In
the method of characteristics which he worked out may be found the
first trace of the Abz\"ahlende Geometrie\footnote{Loria, G.,
Notizie storiche sulla Geometria numerativa. Bibliotheca Mathematica,
1888, pp. 39, 67; 1889, p. 23.} which has been developed by Jonqui\`eres,
Halphen (1875), and Schubert (1876, 1879), and to which Clebsch, Lindemann,
and Hurwitz have also contributed. The general theory of correspondence starts
with Geometry, and Chasles (1864) undertook the first special
researches on the correspondence of algebraic curves, limiting his
investigations, however, to curves of deficiency zero. Cayley (1866)
carried this theory to curves of higher deficiency, and Brill (from
1873) completed the theory.

Cayley's\footnote{Biographical Notice in Cayley's Collected papers,
Vol. VIII.} influence on geometry was very great. He early carried
on Pl\"ucker's consideration of singularities of a curve, and showed
(1864, 1866) that every singularity may be considered as compounded
of ordinary singularities so that the ``six equations'' apply to a
curve with any singularities whatsoever. He thus opened a field for
the later investigations of Noether, Zeuthen, Halphen, and
H.~J.~S.~Smith. Cayley's theorems on the intersection of curves
(1843) and the determination of self-corresponding points for
algebraic correspondences of a simple kind are fundamental in the
present theory, subjects to which Bacharach, Brill, and Noether have
also contributed extensively. Cayley added much to the theories of
rational transformation and correspondence, showing the distinction
between the theory of transformation of spaces and that of
correspondence of loci. His investigations on the bitangents of
plane curves, and in particular on the twenty-eight bitangents of a
non-singular quartic, his developments of Pl\"ucker's conception of
foci, his discussion of the osculating conics of curves and of the
sextactic points on a plane curve, the geometric theory of the
invariants and covariants of plane curves, are all noteworthy. He
was the first to announce (1849) the twenty-seven lines which lie on
a cubic surface, he extended Salmon's theory of reciprocal surfaces,
and treated (1869) the classification of cubic surfaces, a subject
already discussed by Schl\"afli. He also contributed to the theory
of scrolls (skew-ruled surfaces), orthogonal systems of surfaces,
the wave surface, etc., and was the first to reach (1845) any very
general results in the theory of curves of double curvature, a
theory in which the next great advance was made (1882) by Halphen
and Noether. Among Cayley's other contributions to geometry is his
theory of the Absolute, a figure in connection with which all
metrical properties of a figure are considered.

Clebsch\footnote{Klein, Evanston Lectures, Lect. I.} was also
prominent in the study of curves and surfaces. He first applied the
algebra of linear transformation to geometry. He emphasized the idea
of deficiency (Geschlecht) of a curve, a notion which dates back to
Abel, and applied the theory of elliptic and Abelian functions to
geometry, using it for the study of curves. Clebsch (1872)
investigated the shapes of surfaces of the third order. Following
him, Klein attacked the problem of determining all possible forms of
such surfaces, and established the fact that by the principle of
continuity all forms of real surfaces of the third order can be
derived from the particular surface having four real conical
points. Zeuthen (1874) has discussed the various forms of plane
curves of the fourth order, showing the relation between his results
and those of Klein on cubic surfaces. Attempts have been made to
extend the subject to curves of the $n$th order, but no general
classification has been made. Quartic surfaces have been studied by
Rohn (1887) but without a complete enumeration, and the same writer
has contributed (1881) to the theory of Kummer surfaces.

Lie has adopted Pl\"ucker's generalized space element and extended the
theory. His sphere geometry treats the subject from the higher
standpoint of six homogeneous coordinates, as distinguished from the
elementary sphere geometry with but five and characterized by the
conformal group, a geometry studied by Darboux. Lie's theory of
contact transformations, with its application to differential
equations, his line and sphere complexes, and his work on minimum
surfaces are all prominent.

Of great help in the study of curves and surfaces and of the theory
of functions are the models prepared by Dyck, Brill, O. Henrici,
Schwarz, Klein, Sch\"onflies, Kummer, and others.\footnote{Dyck,
W., Katalog mathematischer und mathematisch-physikalischer Modelle,
M\"unchen, 1892; Deutsche Universit\"atsausstellung, Mathematical
Papers of Chicago Congress, p. 49.}

The Theory of Minimum Surfaces has been developed along with the
analytic geometry in general. Lagrange (1760-61) gave the equation
of the minimum surface through a given contour, and Meusnier (1776,
published in 1785) also studied the question. But from this time on
for half a century little that is noteworthy was done, save by
Poisson (1813) as to certain imaginary surfaces. Monge (1784) and
Legendre (1787) connected the study of surfaces with that of
differential equations, but this did not immediately affect this
question. Scherk (1835) added a number of important results, and
first applied the labors of Monge and Legendre to the
theory. Catalan (1842), Bj\"orling (1844), and Dini (1865) have added
to the subject. But the most prominent contributors have been
Bonnet, Schwarz, Darboux, and Weierstrass. Bonnet (from 1853) has
set forth a new system of formulas relative to the general theory of
surfaces, and completely solved the problem of determining the
minimum surface through any curve and admitting in each point of
this curve a given tangent plane, Weierstrass (1866) has contributed
several fundamental theorems, has shown how to find all of the real
algebraic minimum surfaces, and has shown the connection between the
theory of functions of an imaginary variable and the theory of
minimum surfaces.

\chapter{MODERN GEOMETRY.}

Descriptive\footnote{Wiener, Chr., Lehrbuch der darstellenden
Geometrie, Leipzig, 1884-87; Geschichte der darstellenden
Geometrie, 1884.}, Projective, and Modern Synthetic Geometry are so
interwoven in their historic development that it is even more
difficult to separate them from one another than from the analytic
geometry just mentioned. Monge had been in possession of his theory
for over thirty years before the publication of his G\'eom\'etrie
Descriptive (1800), a delay due to the jealous desire of the
military authorities to keep the valuable secret. It is true that
certain of its features can be traced back to Desargues, Taylor,
Lambert, and Fr\'ezier, but it was Monge who worked it out in detail
as a science, although Lacroix (1795), inspired by Monge's lectures
in the \'Ecole Polytechnique, published the first work on the
subject. After Monge's work appeared, Hachette (1812, 1818, 1821)
added materially to its symmetry, subsequent French contributors
being Leroy (1842), Olivier (from 1845), de la Gournerie (from
1860), Vall\'ee, de Fourcy, Adh\'emar, and others. In Germany leading
contributors have been Ziegler (1843), Anger (1858), and especially
Fiedler (3d edn.~1883-88) and Wiener (1884-87). At this period
Monge by no means confined himself to the descriptive geometry. So
marked were his labors in the analytic geometry that he has been
called the father of the modern theory. He also set forth the
fundamental theorem of reciprocal polars, though not in modern
language, gave some treatment of ruled surfaces, and extended the
theory of polars to quadrics.\footnote{On recent development of
graphic methods and the influence of Monge upon this branch of
mathematics, see Eddy, H. T., Modern Graphical Developments,
Mathematical Papers of Chicago Congress (New York, 1896), p 58.}

Monge and his school concerned themselves especially with the
relations of form, and particularly with those of surfaces and
curves in a space of three dimensions. Inspired by the general
activity of the period, but following rather the steps of Desargues
and Pascal, Carnot treated chiefly the metrical relations of
figures. In particular he investigated these relations as connected
with the theory of transversals, a theory whose fundamental property
of a four-rayed pencil goes back to Pappos, and which, though
revived by Desargues, was set forth for the first time in its
general form in Carnot's G\'eom\'etrie de Position (1803), and
supplemented in his Th\'eorie des Transversales (1806). In these
works he introduced negative magnitudes, the general quadrilateral
and quadrangle, and numerous other generalizations of value to the
elementary geometry of to-day. But although Carnot's work was
important and many details are now commonplace, neither the name of
the theory nor the method employed have endured. The present
Geometry of Position (Geometrie der Lage) has little in common with
Carnot's G\'eom\'etrie de Position.

Projective Geometry had its origin somewhat later than the period of
Monge and Carnot. Newton had discovered that all curves of the third
order can be derived by central projection from five fundamental
types. But in spite of this fact the theory attracted so little
attention for over a century that its origin is generally ascribed
to Poncelet. A prisoner in the Russian campaign, confined at
Saratoff on the Volga (1812-14), ``priv\'e,'' as he says, ``de toute
esp\`ece de livres et de secours, surtout distrait par les
malheurs de ma patrie et les miens propres,'' he still had the vigor
of spirit and the leisure to conceive the great work which he
published (1822) eight years later. In this work was first made
prominent the power of central projection in demonstration and the
power of the principle of continuity in research. His leading idea
was the study of projective properties, and as a foundation
principle he introduced the anharmonic ratio, a concept, however,
which dates back to Pappos and which Desargues (1639) had also
used. M\"obius, following Poncelet, made much use of the anharmonic
ratio in his Barycentrische Calc\"ul (1827), but under the name
``Doppelschnitt-Verh\"altniss'' (ratio bisectionalis), a term now in
common use under Steiner's abbreviated form ``Doppelverh\"altniss.''
The name ``anharmonic ratio'' or ``function'' (rapport anharmonique,
or fonction anharmonique) is due to Chasles, and ``cross-ratio'' was
coined by Clifford. The anharmonic point and line properties of
conics have been further elaborated by Brianchon, Chasles, Steiner,
and von Staudt. To Poncelet is also due the theory of ``figures
homologiques,'' the perspective axis and perspective center (called
by Chasles the axis and center of homology), an extension of
Carnot's theory of transversals, and the ``cordes id\'eales'' of
conics which Pl\"ucker applied to curves of all orders, He also
discovered what Salmon has called ``the circular points at
infinity,'' thus completing and establishing the first great
principle of modern geometry, the principle of continuity. Brianchon
(1806), through his application of Desargues's theory of polars,
completed the foundation which Monge had begun for Poncelet's (1829)
theory of reciprocal polars.

Among the most prominent geometers contemporary with Poncelet was
Gergonne, who with more propriety might be ranked as an analytic
geometer. He first (1813) used the term ``polar'' in its modern
geometric sense, although Servois (1811) had used the expression
``pole.'' He was also the first (1825-26) to grasp the idea that
the parallelism which Maurolycus, Snell, and Viete had noticed is a
fundamental principle. This principle he stated and to it he gave
the name which it now bears, the Principle of Duality, the most
important, after that of continuity, in modern geometry. This
principle of geometric reciprocation, the discovery of which was
also claimed by Poncelet, has been greatly elaborated and has found
its way into modern algebra and elementary geometry, and has
recently been extended to mechanics by Genese. Gergonne was the
first to use the word ``class'' in describing a curve, explicitly
defining class and degree (order) and showing the duality between
the two. He and Chasles were among the first to study scientifically
surfaces of higher order.

Steiner (1832) gave the first complete discussion of the projective
relations between rows, pencils, etc., and laid the foundation for
the subsequent development of pure geometry. He practically closed
the theory of conic sections, of the corresponding figures in
three-dimensional space and of surfaces of the second order, and
hence with him opens the period of special study of curves and
surfaces of higher order. His treatment of duality and his
application of the theory of projective pencils to the generation of
conics are masterpieces. The theory of polars of a point in regard
to a curve had been studied by Bobillier and by Grassmann, but
Steiner (1848) showed that this theory can serve as the foundation
for the study of plane curves independently of the use of
coordinates, and introduced those noteworthy curves covariant to a
given curve which now bear the names of himself, Hesse, and Cayley.
This whole subject has been extended by Grassmann, Chasles,
Cremona, and Jonqui\`eres. Steiner was the first to make prominent
(1832) an example of correspondence of a more complicated nature
than that of Poncelet, M\"obius, Magnus, and Chasles. His
contributions, and those of Gudermann, to the geometry of the sphere
were also noteworthy.

While M\"obius, Pl\"ucker, and Steiner were at work in Germany, Chasles
was closing the geometric era opened in France by Monge. His Aper\c{c}u
Historique (1837) is a classic, and did for France what Salmon's
works did for algebra and geometry in England, popularizing the
researches of earlier writers and contributing both to the theory
and the nomenclature of the subject. To him is due the name
``homographic'' and the complete exposition of the principle as
applied to plane and solid figures, a subject which has received
attention in England at the hands of Salmon, Townsend, and
H.~J.~S.~Smith.

Von Staudt began his labors after Pl\"ucker, Steiner, and Chasles had
made their greatest contributions, but in spite of this seeming
disadvantage he surpassed them all. Joining the Steiner school, as
opposed to that of Pl\"ucker, he became the greatest exponent of pure
synthetic geometry of modern times. He set forth (1847, 1856-60) a
complete, pure geometric system in which metrical geometry finds no
place. Projective properties foreign to measurements are
established independently of number relations, number being drawn
from geometry instead of conversely, and imaginary elements being
systematically introduced from the geometric side. A projective
geometry based on the group containing all the real projective and
dualistic transformations, is developed, imaginary transformations
being also introduced. Largely through his influence pure geometry
again became a fruitful field. Since his time the distinction
between the metrical and projective theories has been to a great
extent obliterated,\footnote{Klein, F., Erlangen Programme of
1872, Haskell's translation, Bulletin of New York Mathematical
Society, Vol. II, p. 215.} the metrical properties being considered
as projective relations to a fundamental configuration, the circle
at infinity common for all spheres. Unfortunately von Staudt wrote
in an unattractive style, and to Reye is due much of the popularity
which now attends the subject.

Cremona began his publications in 1862. His elementary work on
projective geometry (1875) in Leudesdorf's translation is familiar
to English readers. His contributions to the theory of geometric
transformations are valuable, as also his works on plane curves,
surfaces, etc.

In England Mulcahy, but especially Townsend (1863), and Hirst, a
pupil of Steiner's, opened the subject of modern geometry. Clifford
did much to make known the German theories, besides himself
contributing to the study of polars and the general theory of curves.

\chapter{ELEMENTARY GEOMETRY.}

Trigonometry and Elementary Geometry have also been affected by the
general mathematical spirit of the century. In trigonometry the
general substitution of ratios for lines in the definitions of
functions has simplified the treatment, and certain formulas have
been improved and others added.\footnote{Todhunter, I., History of
certain formulas of spherical trigonometry, Philosophical Magazine,
1873.} The convergence of trigonometric series, the introduction of
the Fourier series, and the free use of the imaginary have already
been mentioned. The definition of the sine and cosine by series, and
the systematic development of the theory on this basis, have been
set forth by Cauchy (1821), Lobachevsky (1833), and others. The
hyperbolic trigonometry,\footnote{Gunther, S., Die Lehre von den
gew\"ohnlichen und verallgemeinerten Hyperbelfunktionen, Halle, 1881;
Chrystal, G., Algebra, Vol. II, p. 288.} already founded by Mayer and
Lambert, has been popularized and further developed by Gudermann
(1830), Ho\"uel, and Laisant (1871), and projective formulas and
generalized figures have been introduced, notably by Gudermann,
M\"obius, Poncelet, and Steiner. Recently Study has investigated the
formulas of spherical trigonometry from the standpoint of the modern
theory of functions and theory of groups, and Macfarlane has
generalized the fundamental theorem of trigonometry for
three-dimensional space.

Elementary Geometry has been even more affected. Among the many
contributions to the theory may be mentioned the following: That of
M\"obius on the opposite senses of lines, angles, surfaces, and
solids; the principle of duality as given by Gergonne and Poncelet;
the contributions of De Morgan to the logic of the subject; the
theory of transversals as worked out by Monge, Brianchon, Servois,
Carnot, Chasles, and others; the theory of the radical axis, a
property discovered by the Arabs, but introduced as a definite
concept by Gaultier (1813) and used by Steiner under the name of
``line of equal power''; the researches of Gauss concerning
inscriptible polygons, adding the 17- and 257-gon to the list below
the 1000-gon; the theory of stellar polyhedra as worked out by
Cauchy, Jacobi, Bertrand, Cayley, M\"obius, Wiener, Hess, Hersel,
and others, so that a whole series of bodies have been added to the
four Kepler-Poinsot regular solids; and the researches of Muir on
stellar polygons. These and many other improvements now find more or
less place in the text-books of the day.

To these must be added the recent Geometry of the Triangle, now a
prominent chapter in elementary mathematics. Crelle (1816) made
some investigations in this line, Feuerbach (1822) soon after
discovered the properties of the Nine-Point Circle, and Steiner also
came across some of the properties of the triangle, but none of
these followed up the investigation. Lemoine\footnote{Smith,
D. E., Biography of Lemoine, American Mathematical Monthly,
Vol. III, p. 29; Mackay, J. S., various articles on modern geometry
in Proceedings Edinburgh Mathematical Society, various years;
Vigari\'e, \'E., G\'eom\'etrie du triangle. Articles in recent
numbers of Journal de Math\'ematiques sp\'eciales, Mathesis, and
Proceedings of the Association fran\c{c}aise pour l'avancement des
sciences.} (1873) was the first to take up the subject in a
systematic way, and he has contributed extensively to its
development. His theory of ``transformation continue'' and his
``g\'eom\'etrographie'' should also be mentioned. Brocard's
contributions to the geometry of the triangle began in 1877. Other
prominent writers have been Tucker, Neuberg, Vigari\'e, Emmerich,
M'Cay, Longchamps, and H. M. Taylor. The theory is also greatly
indebted to Miller's work in The Educational Times, and to
Hoffmann's Zeitschrift.

The study of linkages was opened by Peaucellier (1864), who gave the
first theoretically exact method for drawing a straight line. Kempe
and Sylvester have elaborated the subject.

In recent years the ancient problems of trisecting an angle,
doubling the cube, and squaring the circle have all been settled by
the proof of their insolubility through the use of compasses and
straight edge.\footnote{Klein, F., Vortr\"age \"uber ausgew\"ahlten
Fragen; Rudio, F., Das Problem von der Quadratur des Zirkels.
Naturforschende Gesellschaft Vierteljahrschrift, 1890; Archimedes,
Huygens, Lambert, Legendre (Leipzig, 1892).}

\chapter{NON-EUCLIDEAN GEOMETRY.}

The Non-Euclidean Geometry\footnote{St\"ackel and Engel, Die
Theorie der Parallellinien von Euklid bis auf Gauss, Leipzig, 1895;
Halsted, G. B., various contributions: Bibliography of Hyperspace
and Non-Euclidean Geometry, American Journal of Mathematics,
Vols. I, II; The American Mathematical Monthly, Vol. I; translations
of Lobachevsky's Geometry, Vasiliev's address on Lobachevsky,
Saccheri's Geometry, Bolyai's work and his life; Non-Euclidean and
Hyperspaces, Mathematical Papers of Chicago Congress, p. 92. Loria,
G., Die haupts\"achlichsten Theorien der Geometrie, p. 106;
Karagiannides, A., Die Nichteuklidische Geometrie vom Alterthum bis
zur Gegenwart, Berlin, 1893; McClintock, E., On the early history of
Non-Euclidean Geometry, Bulletin of New York Mathematical Society,
Vol. II, p. 144; Poincar\'e, Non-Euclidean Geom., Nature, 45:404;
Articles on Parallels and Measurement in Encyclop\ae{}dia Britannica,
9th edition; Vasiliev's address (German by Engel) also appears in
the Abhandlungen zur Geschichte der Mathematik, 1895.} is a natural
result of the futile attempts which had been made from the time of
Proklos to the opening of the nineteenth century to prove the fifth
postulate (also called the twelfth axiom, and sometimes the eleventh
or thirteenth) of Euclid. The first scientific investigation of
this part of the foundation of geometry was made by Saccheri (1733),
a work which was not looked upon as a precursor of Lobachevsky,
however, until Beltrami (1889) called attention to the fact. Lambert
was the next to question the validity of Euclid's postulate, in his
Theorie der Parallellinien (posthumous, 1786), the most important of
many treatises on the subject between the publication of Saccheri's
work and those of Lobachevsky and Bolyai. Legendre also worked in
the field, but failed to bring himself to view the matter outside
the Euclidean limitations.

During the closing years of the eighteenth century
Kant's\footnote{Fink, E., Kant als Mathematiker, Leipzig, 1889.}
doctrine of absolute space, and his assertion of the necessary
postulates of geometry, were the object of much scrutiny and
attack. At the same time Gauss was giving attention to the fifth
postulate, though on the side of proving it. It was at one time
surmised that Gauss was the real founder of the non-Euclidean
geometry, his influence being exerted on Lobachevsky through his
friend Bartels, and on Johann Bolyai through the father Wolfgang,
who was a fellow student of Gauss's. But it is now certain that
Gauss can lay no claim to priority of discovery, although the
influence of himself and of Kant, in a general way, must have had
its effect.

Bartels went to Kasan in 1807, and Lobachevsky was his pupil. The
latter's lecture notes show that Bartels never mentioned the subject
of the fifth postulate to him, so that his investigations, begun
even before 1823, were made on his own motion and his results were
wholly original. Early in 1826 he sent forth the principles of his
famous doctrine of parallels, based on the assumption that through a
given point more than one line can be drawn which shall never meet a
given line coplanar with it. The theory was published in full in
1829-30, and he contributed to the subject, as well as to other
branches of mathematics, until his death.

Johann Bolyai received through his father, Wolfgang, some of the
inspiration to original research which the latter had received from
Gauss. When only twenty-one he discovered, at about the same time as
Lobachevsky, the principles of non-Euclidean geometry, and refers to
them in a letter of November, 1823. They were committed to writing
in 1825 and published in 1832. Gauss asserts in his correspondence
with Schumacher (1831-32) that he had brought out a theory along the
same lines as Lobachevsky and Bolyai, but the publication of their
works seems to have put an end to his investigations. Schweikart
was also an independent discoverer of the non-Euclidean geometry, as
his recently recovered letters show, but he never published anything
on the subject, his work on the theory of parallels (1807), like
that of his nephew Taurinus (1825), showing no trace of the
Lobachevsky-Bolyai idea.

The hypothesis was slowly accepted by the mathematical world. Indeed
it was about forty years after its publication that it began to
attract any considerable attention. Ho\"uel (1866) and Flye
St. Marie (1871) in France, Riemann (1868), Helmholtz (1868),
Frischauf (1872), and Baltzer (1877) in Germany, Beltrami (1872) in
Italy, de Tilly (1879) in Belgium, Clifford in England, and Halsted
(1878) in America, have been among the most active in making the
subject popular. Since 1880 the theory may be said to have become
generally understood and accepted as legitimate.\footnote{For an
excellent summary of the results of the hypothesis, see an article
by McClintock, The Non-Euclidian Geometry, Bulletin of New York
Mathematical Society, Vol. II, p. 1.}

Of all these contributions the most noteworthy from the scientific
standpoint is that of Riemann. In his Habilitationsschrift (1854)
he applied the methods of analytic geometry to the theory, and
suggested a surface of negative curvature, which Beltrami calls
``pseudo-spherical,'' thus leaving Euclid's geometry on a surface of
zero curvature midway between his own and Lobachevsky's. He thus set
forth three kinds of geometry, Bolyai having noted only two. These
Klein (1871) has called the elliptic (Riemann's), parabolic
(Euclid's), and hyperbolic (Lobachevsky's).

Starting from this broader point of view\footnote{Klein. Evanston
Lectures. Lect. IX.} there have contributed to the subject many of
the leading mathematicians of the last quarter of a century,
including, besides those already named, Cayley, Lie, Klein, Newcomb,
Pasch, C.~S.~Peirce, Killing, Fiedler, Mansion, and
McClintock. Cayley's contribution of his ``metrical geometry'' was
not at once seen to be identical with that of Lobachevsky and
Bolyai. It remained for Klein (1871) to show this, thus simplifying
Cayley's treatment and adding one of the most important results of
the entire theory. Cayley's metrical formulas are, when the Absolute
is real, identical with those of the hyperbolic geometry; when it
is imaginary, with the elliptic; the limiting case between the two
gives the parabolic (Euclidean) geometry. The question raised by
Cayley's memoir as to how far projective geometry can be defined in
terms of space without the introduction of distance had already been
discussed by von Staudt (1857) and has since been treated by Klein
(1873) and by Lindemann (1876).

\backmatter

\chapter{BIBLIOGRAPHY.}

%% In the book, the titles are slightly smaller then the rest
%% of the text; should we follow that here?

The following are a few of the general works on the history of
mathematics in the nineteenth century, not already mentioned in the
foot-notes. For a complete bibliography of recent works the reader
should consult the Jahrbuch \"uber die Fortschritte der Mathematik,
the Bibliotheca Mathematica, or the Revue Semestrielle, mentioned
below.

\bigskip
Abhandlungen zur Geschichte der Mathematik (Leipzig).

Ball, W.~W.~R., A short account of the history of mathematics
(London, 1893).

Ball, W.~W.~R., History of the study of mathematics at Cambridge
(London, 1889).

Ball, W.~W.~R., Primer of the history of mathematics (London, 1895).

Bibliotheca Mathematica, G. Enestr\"om, Stockholm. Quarterly.
Should be consulted for bibliography of current articles and works
on history of mathematics.

Bulletin des Sciences Math\'ematiques (Paris, II\up{i\`eme} Partie).

Cajori, F., History of Mathematics (New York, 1894).

Cayley, A., Inaugural address before the British Association,
1883. Nature, Vol. XXVIII, p. 491.

Dictionary of National Biography. London, not completed. Valuable
on biographies of British mathematicians.

D'Ovidio, Enrico, Uno sguardo alle origini ed allo sviluppo della
Matematica Pura (Torino, 1889).

Dupin, Ch., Coup d'\oe{}il sur quelques progr\`es des Sciences
math\'ematiques, en France, 1830-35. Comptes Rendus, 1835.

Encyclop\ae{}dia Britannica. Valuable biographical articles by Cayley,
Chrystal, Clerke, and others.

Fink, K., Geschichte der Mathematik (T\"ubingen, 1890). Bibliography
on p. 255.

Gerhardt, C.~J., Geschichte der Mathematik in Deutschland (Munich,
1877).

Graf, J.~H., Geschichte der Mathematik und der Naturwissenschaften
in bernischen Landen (Bern, 1890). Also numerous biographical
articles.

G\"unther, S., Vermischte Untersuchungen zur Geschichte der
mathematischen Wissenschaften (Leipzig, 1876).

G\"unther, S., Ziele und Resultate der neueren
mathematisch-historischen Forschung (Erlangen, 1876).

Hagen, J.~G., Synopsis der h\"oheren Mathematik. Two volumes
(Berlin, 1891-93).

Hankel, H., Die Entwickelung der Mathematik in dem letzten
Jahrhundert (T\"ubingen, 1884).

Hermite, Ch., Discours prononc\'e devant le pr\'esident de la
r\'epublique le 5 ao\^ut 1889 \`a l'inauguration de la nouvelle
Sorbonne. Bulletin des Sciences math\'ematiques, 1890; also Nature,
Vol. XLI, p. 597. (History of nineteenth-century mathematics in
France.)

Hoefer, F., Histoire des math\'ematiques (Paris, 1879).

Isely, L., Essai sur l'histoire des math\'ematiques dans la Suisse
fran\c{c}aise (Neuch\^atel, 1884).

Jahrbuch \"uber die Fortschritte der Mathematik (Berlin, annually,
1868 to date).

Marie, M., Histoire des sciences math\'ematiques et physiques.
Vols. X, XI, XII (Paris, 1887-88).

Matthiessen, L., Grundz\"uge der antiken und modernen Algebra der
litteralen Gleichungen (Leipzig, 1878).

Newcomb, S., Modern mathematical thought. Bulletin New York
Mathematical Society, Vol. III, p. 95; Nature, Vol. XLIX, p. 325.

Poggendorff, J.~C., Biographisch-literarisches Handw\"orterbuch
zur Ge\-schi\-chte der exacten Wissenschaften. Two volumes (Leipzig,
1863), and two later supplementary volumes.

Quetelet, A., Sciences math\'ematiques et physiques chez les Belges
au commencement du XIX\up{e} si\`ecle (Brussels, 1866).

Revue semestrielle des publications math\'ematiques r\'edig\'ee sous
les auspices de la Soci\'et\'e math\'ematique d'Amsterdam. 1893 to
date. (Current periodical literature.)

Roberts, R.~A., Modern mathematics. Proceedings of the Irish
Academy, 1888.

Smith, H.~J.~S., On the present state and prospects of some branches
of pure mathematics. Proceedings of London Mathematical Society,
1876; Nature, Vol. XV, p. 79.

Sylvester, J.~J., Address before the British Association. Nature,
Vol. I, pp. 237, 261.

Wolf, R., Handbuch der Mathematik. Two volumes (Zurich, 1872).

Zeitschrift f\"ur Mathematik und Physik. Historisch-literarische
Abtheilung. Leipzig. The Abhandlungen zur Geschichte der Mathematik
are supplements.

\bigskip

For a biographical table of mathematicians see Fink's Geschichte der
Mathematik, p. 240. For the names and positions of living
mathematicians see the Jahrbuch der gelehrten Welt, published at
Strassburg.

Since the above bibliography was prepared the nineteenth century has
closed. With its termination there would naturally be expected a
series of retrospective views of the development of a hundred years
in all lines of human progress. This expectation was duly
fulfilled, and numerous addresses and memoirs testify to the
interest recently awakened in the subject. Among the contributions
to the general history of modern mathematics may be cited the
following:

\bigskip
Pierpont, J., St. Louis address, 1904. Bulletin of the American
Mathematical Society (N. S.), Vol. IX, p. 136. An excellent survey
of the century's progress in pure mathematics.

G\"unther, S., Die Mathematik im neunzehnten Jahrhundert. Hoffmann's
Zeitschrift, Vol. XXXII, p. 227.

Adh\'emar, R. d', L'\oe{}uvre math\'ematique du XIX\up{e} si\`ecle. Revue
des questions scientifiques, Louvain Vol. XX (2), p. 177 (1901).

Picard, E., Sur le d\'eveloppement, depuis un si\`ecle, de quelques
th\'eories fondamentales dans l'analyse
math\'ematique. Conf\'erences faite \`a Clark University (Paris,
1900).

Lampe, E., Die reine Mathematik in den Jahren 1884-1899 (Berlin,
1900).

\bigskip

Among the contributions to the history of applied mathematics in
general may be mentioned the following:

\bigskip

Woodward, R.~S., Presidential address before the American
Mathematical Society in December, 1899. Bulletin of the American
Mathematical Society (N. S.), Vol. VI, p. 133. (German, in the
Naturwiss. Rundschau, Vol. XV; Polish, in the Wiadomo\'sci
Matematyczne, Warsaw, Vol. V (1901).). This considers the century's
progress in applied mathematics.

Mangoldt, H. von, Bilder aus der Entwickelung der reinen und
angewandten Mathematik w\"ahrend des neunzehnten Jahrhunderts mit
besonderer Ber\"ucksichtigung des Einflusses von Carl Friedrich
Gauss. Festrede (Aachen, 1900).

Van t' Hoff, J.~H., Ueber die Entwickelung der exakten
Naturwissenschaften im 19. Jahrhundert. Vortrag gehalten in Aachen,
1900. Naturwiss. Rundschau, Vol. XV, p. 557 (1900).

\bigskip

The following should be mentioned as among the latest contributions
to the history of modern mathematics in particular countries:

\bigskip

Fiske, T.~S., Presidential address before the American Mathematical
Society in December, 1904. Bulletin of the American Mathematical
Society (N. S.), Vol. IX, p. 238. This traces the development of
mathematics in the United States.

Purser, J., The Irish school of mathematicians and physicists from
the beginning of the nineteenth century. Nature, Vol. LXVI, p. 478
(1902).

Guimar\~aes, R. Les math\'ematiques en Portugal au XIX\up{e} si\`ecle.
(Co\"{\i}mbre, 1900).

\bigskip

A large number of articles upon the history of special branches of
mathematics have recently appeared, not to mention the custom of
inserting historical notes in the recent treatises upon the subjects
themselves. Of the contributions to the history of particular
branches, the following may be mentioned as types:

\bigskip

Miller, G.~A., Reports on the progress in the theory of groups of a
finite order. Bulletin of the American Mathematical Society (N. S.),
Vol. V, p. 227; Vol. IX, p. 106. Supplemental report by Dickson,
L. E., Vol. VI, p. 13, whose treatise on Linear Groups (1901) is a
history in itself. Steinitz and Easton have also contributed to this
subject.

Hancock, H., On the historical development of the Abelian functions
to the time of Riemann. British Association Report for 1897.

Brocard, H., Notes de bibliographie des courbes g\'eom\'etriques.
Bar-le-Duc, 2 vols., lithog., 1897, 1899.

Hagen, J.~G., On the history of the extensions of the calculus.
Bulletin of the American Mathematical Society (N. S.), Vol. VI,
p. 381.

Hill, J.~E., Bibliography of surfaces and twisted curves. Ib., Vol.
III, p. 133 (1897).

Aubry, A., Historia del problema de las tangentes. El Progresso
matematico, Vol. I (2), pp. 129, 164.

Comp\`ere, C., Le probl\`eme des brachistochrones. Essai historique.
M\'emoires de la Soci\'et\'e d. Sciences, Li\`ege, Vol. I (3),
p. 128 (1899).

St\"ackel, P., Beitr\"age zur Geschichte der Funktionentheorie im
achtzehnten Jahrhundert. Bibliotheca Mathematica, Vol. II (3),
p. 111 (1901).

Obenrauch, F.~J., Geschichte der darstellenden und projektiven
Geometrie mit besonderer Ber\"ucksichtigung ihrer Begr\"undung in
Frankreich und Deutschiand und ihrer wissenschaftlichen Pflege in
Oesterreich (Br\"unn, 1897).

Muir, Th., The theory of alternants in the historical order of its
development up to 1841. Proceedings of the Royal Society of
Edinburgh, Vol. XXIII (2), p. 93 (1899). The theory of screw
determinants and Pfaffians in the historical order of its
development up to 1857. Ib., p. 181.

Papperwitz, E., Ueber die wissenschaftliche Bedeutung der
darstellenden Geometrie und ihre Entwickelung bis zur
systematischen Begr\"undung durch Gaspard Monge. Rede (Freiberg
i./S., 1901).

\bigskip

Mention should also be made of the fact that the Bibliotheca
Mathematica, a journal devoted to the history of the mathematical
sciences, began its third series in 1900. It remains under the able
editorship of G. Enestr\"om, and in its new series it appears in
much enlarged form. It contains numerous articles on the history of
modern mathematics, with a complete current bibliography of this
field.

Besides direct contributions to the history of the subject, and
historical and bibliographical notes, several important works have
recently appeared which are historical in the best sense, although
written from the mathematical standpoint. Of these there are three
that deserve special mention:

\bigskip

Encyklop\"adie der mathematischen Wissenschaften mit Einschluss
ihrer Anwendungen. The publication of this monumental work was begun
in 1898, and the several volumes are being carried on
simultaneously. The first volume (Arithmetik and Algebra) was
completed in 1904. This publication is under the patronage of the
academies of sciences of G\"ottingen, Leipzig, Munich, and Vienna. A
French translation, with numerous additions, is in progress.

Pascal, E., Repertorium der h\"oheren Mathematik, translated from
the Italian by A. Schepp. Two volumes (Leipzig, 1900, 1902). It
contains an excellent bibliography, and is itself a history of
modern mathematics.

Hagen, J. G., Synopsis der h\"oheren Mathematik. This has been for
some years in course of publication, and has now completed Vol. III.

\bigskip

In the line of biography of mathematicians, with lists of published
works, Poggendorff's Biographisch-literarisches Handw\"orterbuch zur
Geschichte der exacten Wissenschaften has reached its fourth volume
(Leipzig, 1903), this volume covering the period from 1883 to
1902. A new biographical table has been added to the English
translation of Fink's History of Mathematics (Chicago, 1900).

\chapter{GENERAL TENDENCIES.}

The opening of the nineteenth century was, as we have seen, a period
of profound introspection following a period of somewhat careless
use of the material accumulated in the seventeenth century. The
mathematical world sought to orientate itself, to examine the
foundations of its knowledge, and to critically examine every step
in its several theories. It then took up the line of discovery once
more, less recklessly than before, but still with thoughts directed
primarily in the direction of invention. At the close of the
century there came again a period of introspection, and one of the
recent tendencies is towards a renewed study of foundation
principles. In England one of the leaders in this movement is
Russell, who has studied the foundations of geometry (1897) and of
mathematics in general (1903). In America the fundamental
conceptions and methods of mathematics have been considered by
B\^ocher in his St. Louis address in 1904,\footnote{Bulletin of the
American Mathematical Society (N. S.), Vol. XI, p. 115.} and the
question of a series of irreducible postulates has been studied by
Huntington. In Italy, Padoa and Bureli-Forti have studied the
fundamental postulates of algebra, and Pieri those of geometry. In
Germany, Hilbert has probably given the most attention to the
foundation principles of geometry (1899), and more recently he has
investigated the compatibility of the arithmetical axioms (1900). In
France, Poincar\'e has considered the r\^ole of intuition and of
logic in mathematics,\footnote{Compte rendu du deuxi\`eme congr\`es
international des math\'ematiciens tenu \`a Paris, 1900. Paris,
1902, p. 115.} and in every country the foundation principles have
been made the object of careful investigation.

As an instance of the orientation already mentioned, the noteworthy
address of Hilbert at Paris in 1900\footnote{G\"ottinger
Nachrichten, 1900, p. 253; Archiv der Mathematik und Physik, 1901,
pp. 44, 213; Bulletin of the American Mathematical Society, 1902,
p. 437.} stands out prominently. This address reviews the field of
pure mathematics and sets forth several of the greatest questions
demanding investigation at the present time. In the particular line
of geometry the memoir which Segr\'e wrote in 1891, on the
tendencies in geometric investigation, has recently been revised and
brought up to date.\footnote{Bulletin of the American Mathematical
Society (N. S.), Vol. X, p. 443.}

There is also seen at the present time, as never before, a
tendency to co\"operate, to exchange views, and to internationalize
mathematics. The first international congress of mathematicians
was held at Zurich in 1897, the second one at Paris in 1900, and
the third at Heidelberg in 1904. The first international congress
of philosophy was held at Paris in 1900, the third section
being devoted to logic and the history of the sciences (on this
occasion chiefly mathematics), and the second congress was
held at Geneva in 1904. There was also held an international
congress of historic sciences at Rome in 1903, an international
committee on the organization of a congress on the history of
sciences being at that time formed. The result of such gatherings
has been an exchange of views in a manner never before
possible, supplementing in an inspiring way the older form of
international communication through published papers.

In the United States there has been shown a similar tendency
to exchange opinions and to impart verbal information
as to recent discoveries. The American Mathematical Society,
founded in 1894,\footnote{It was founded as the New York
Mathematical Society six years earlier, in 1888.} has doubled
its membership in the past decade,\footnote{It is now, in 1905,
approximately 500.}
and has increased its average of annual papers from 30 to 150.
It has also established two sections, one at Chicago (1897) and
one at San Francisco (1902). The activity of its members and
the quality of papers prepared has led to the publication of the
\emph{Transactions}, beginning with 1900. In order that its members
may be conversant with the lines of investigation in the various
mathematical centers, the society publishes in its \emph{Bulletin} the
courses in advanced mathematics offered in many of the leading
universities of the world. Partly as a result of this activity,
and partly because of the large number of American students
who have recently studied abroad, a remarkable change is at
present passing over the mathematical work done in the universities
and colleges of this country. Courses that a short time ago
were offered in only a few of our leading universities are now
not uncommon in institutions of college rank. They are often
given by men who have taken advanced degrees in mathematics,
at G\"ottingen, Berlin, Paris, or other leading universities abroad,
and they are awakening a great interest in the modern field.
A recent investigation (1903) showed that 67 students in ten
American institutions were taking courses in the theory of functions,
11 in the theory of elliptic functions, 94 in projective geometry,
26 in the theory of invariants, 45 in the theory of groups,
and 46 in the modern advanced theory of equations, courses
which only a few years ago were rarely given in this country.
A similar change is seen in other countries, notably in England
and Italy, where courses that a few years ago were offered only
in Paris or in Germany are now within the reach of university
students at home. The interest at present manifested by American
scholars is illustrated by the fact that only four countries (Germany,
Russia, Austria, and France) had more representatives
than the United States, among the 336 regular members at the
third international mathematical congress at Heidelberg in 1904.

The activity displayed at the present time in putting the
work of the masters into usable form, so as to define clear points
of departure along the several lines of research, is seen in the
large number of collected works published or in course of publication
in the last decade. These works have usually been
published under governmental patronage, often by some learned
society, and always under the editorship of some recognized
authority. They include the works of Galileo, Fermat, Descartes,
Huygens, Laplace, Gauss, Galois, Cauchy, Hesse, Pl\"ucker,
Grassmann, Dirichlet, Laguerre, Kronecker, Fuchs, Weierstrass,
Stokes, Tait, and various other leaders in mathematics. It is
only natural to expect a number of other sets of collected works
in the near future, for not only is there the remote past to draw
upon, but the death roll of the last decade has been a large one.
The following is only a partial list of eminent mathematicians
who have recently died, and whose collected works have been
or are in the course of being published, or may be deemed worthy
of publication in the future: Cayley (1895), Neumann (1895),
Tisserand (1896), Brioschi (1897), Sylvester (1897), Weierstrass
(1897), Lie (1899), Beltrami (1900), Bertrand (1900), Tait (1901),
Hermite (1901), Fuchs (1902), Gibbs (1903), Cremona (1903),
and Salmon (1904), besides such writers as Frost (1898), Hoppe
(1900), Craig (1900), Schl\"omilch (1901), Everett on the side of
mathematical physics (1904), and Paul Tannery, the best of
the modern French historians of mathematics (1904).\footnote{For
students wishing to investigate the work of mathematicians who died
in the last two decades of the nineteenth century, Enestr\"om's "Bio-bibliographie
der 1881-1900 verstorbenen Mathematiker," in the Bibliotheca Mathematica
Vol. II (3), p. 326 (1901), will be found valuable.}

It is, of course, impossible to detect with any certainty the
present tendencies in mathematics. Judging, however, by the
number and nature of the published papers and works of the
past few years, it is reasonable to expect a great development in
all lines, especially in such modern branches as the theory of
groups, theory of functions, theory of invariants, higher geometry,
and differential equations. If we may judge from the works in
applied mathematics which have recently appeared, we are
entering upon an era similar to that in which Laplace labored,
an era in which all these modern theories of mathematics shall
find application in the study of physical problems, including
those that relate to the latest discoveries. The profound study
of applied mathematics in France and England, the advanced
work in discovery in pure mathematics in Germany and France,
and the search for the logical bases for the science that has distinguished
Italy as well as Germany, are all destined to affect the
character of the international mathematics of the immediate
future. Probably no single influence will be more prominent
in the internationalizing process than the tendency of the younger
generation of American mathematicians to study in England,
France, Germany, and Italy, and to assimilate the best that each
of these countries has to offer to the world.

\newpage



\chapter{PROJECT GUTENBERG "SMALL PRINT"}
\small
\pagenumbering{gobble}
\begin{verbatim}


END OF THE PROJECT GUTENBERG EBOOK HISTORY OF MODERN MATHEMATICS

This file should be named 8746-t.tex or 8746-t.zip

Produced by David Starner, John Hagerson,
and the Online Distributed Proofreading Team

Project Gutenberg eBooks are often created from several printed
editions, all of which are confirmed as Public Domain in the US
unless a copyright notice is included.  Thus, we usually do not
keep eBooks in compliance with any particular paper edition.

We are now trying to release all our eBooks one year in advance
of the official release dates, leaving time for better editing.
Please be encouraged to tell us about any error or corrections,
even years after the official publication date.

Please note neither this listing nor its contents are final til
midnight of the last day of the month of any such announcement.
The official release date of all Project Gutenberg eBooks is at
Midnight, Central Time, of the last day of the stated month.  A
preliminary version may often be posted for suggestion, comment
and editing by those who wish to do so.

Most people start at our Web sites at:
http://gutenberg.net or
http://promo.net/pg

These Web sites include award-winning information about Project
Gutenberg, including how to donate, how to help produce our new
eBooks, and how to subscribe to our email newsletter (free!).


Those of you who want to download any eBook before announcement
can get to them as follows, and just download by date.  This is
also a good way to get them instantly upon announcement, as the
indexes our cataloguers produce obviously take a while after an
announcement goes out in the Project Gutenberg Newsletter.

http://www.ibiblio.org/gutenberg/etext05 or

ftp://ftp.ibiblio.org/pub/docs/books/gutenberg/etext05

Or /etext02, 01, 00, 99, 98, 97, 96, 95, 94, 93, 92, 92, 91 or 90

Just search by the first five letters of the filename you want,
as it appears in our Newsletters.


Information about Project Gutenberg (one page)

We produce about two million dollars for each hour we work.  The
time it takes us, a rather conservative estimate, is fifty hours
to get any eBook selected, entered, proofread, edited, copyright
searched and analyzed, the copyright letters written, etc.   Our
projected audience is one hundred million readers.  If the value
per text is nominally estimated at one dollar then we produce $2
million dollars per hour in 2002 as we release over 100 new text
files per month:  1240 more eBooks in 2001 for a total of 4000+
We are already on our way to trying for 2000 more eBooks in 2002
If they reach just 1-2\% of the world's population then the total
will reach over half a trillion eBooks given away by year's end.

The Goal of Project Gutenberg is to Give Away 1 Trillion eBooks!
This is ten thousand titles each to one hundred million readers,
which is only about 4\% of the present number of computer users.

Here is the briefest record of our progress (* means estimated):

eBooks Year Month

    1  1971 July
   10  1991 January
  100  1994 January
 1000  1997 August
 1500  1998 October
 2000  1999 December
 2500  2000 December
 3000  2001 November
 4000  2001 October/November
 6000  2002 December*
 9000  2003 November*
10000  2004 January*


The Project Gutenberg Literary Archive Foundation has been created
to secure a future for Project Gutenberg into the next millennium.

We need your donations more than ever!

As of February, 2002, contributions are being solicited from people
and organizations in: Alabama, Alaska, Arkansas, Connecticut,
Delaware, District of Columbia, Florida, Georgia, Hawaii, Illinois,
Indiana, Iowa, Kansas, Kentucky, Louisiana, Maine, Massachusetts,
Michigan, Mississippi, Missouri, Montana, Nebraska, Nevada, New
Hampshire, New Jersey, New Mexico, New York, North Carolina, Ohio,
Oklahoma, Oregon, Pennsylvania, Rhode Island, South Carolina, South
Dakota, Tennessee, Texas, Utah, Vermont, Virginia, Washington, West
Virginia, Wisconsin, and Wyoming.

We have filed in all 50 states now, but these are the only ones
that have responded.

As the requirements for other states are met, additions to this list
will be made and fund raising will begin in the additional states.
Please feel free to ask to check the status of your state.

In answer to various questions we have received on this:

We are constantly working on finishing the paperwork to legally
request donations in all 50 states.  If your state is not listed and
you would like to know if we have added it since the list you have,
just ask.

While we cannot solicit donations from people in states where we are
not yet registered, we know of no prohibition against accepting
donations from donors in these states who approach us with an offer to
donate.

International donations are accepted, but we don't know ANYTHING about
how to make them tax-deductible, or even if they CAN be made
deductible, and don't have the staff to handle it even if there are
ways.

Donations by check or money order may be sent to:

Project Gutenberg Literary Archive Foundation
PMB 113
1739 University Ave.
Oxford, MS 38655-4109

Contact us if you want to arrange for a wire transfer or payment
method other than by check or money order.

The Project Gutenberg Literary Archive Foundation has been approved by
the US Internal Revenue Service as a 501(c)(3) organization with EIN
[Employee Identification Number] 64-622154.  Donations are
tax-deductible to the maximum extent permitted by law.  As fund-raising
requirements for other states are met, additions to this list will be
made and fund-raising will begin in the additional states.

We need your donations more than ever!

You can get up to date donation information online at:

http://www.gutenberg.net/donation.html


***

If you can't reach Project Gutenberg,
you can always email directly to:

Michael S. Hart <hart@pobox.com>

Prof. Hart will answer or forward your message.

We would prefer to send you information by email.


**The Legal Small Print**


(Three Pages)

***START**THE SMALL PRINT!**FOR PUBLIC DOMAIN EBOOKS**START***
Why is this "Small Print!" statement here? You know: lawyers.
They tell us you might sue us if there is something wrong with
your copy of this eBook, even if you got it for free from
someone other than us, and even if what's wrong is not our
fault. So, among other things, this "Small Print!" statement
disclaims most of our liability to you. It also tells you how
you may distribute copies of this eBook if you want to.

*BEFORE!* YOU USE OR READ THIS EBOOK
By using or reading any part of this PROJECT GUTENBERG-tm
eBook, you indicate that you understand, agree to and accept
this "Small Print!" statement. If you do not, you can receive
a refund of the money (if any) you paid for this eBook by
sending a request within 30 days of receiving it to the person
you got it from. If you received this eBook on a physical
medium (such as a disk), you must return it with your request.

ABOUT PROJECT GUTENBERG-TM EBOOKS
This PROJECT GUTENBERG-tm eBook, like most PROJECT GUTENBERG-tm eBooks,
is a "public domain" work distributed by Professor Michael S. Hart
through the Project Gutenberg Association (the "Project").
Among other things, this means that no one owns a United States copyright
on or for this work, so the Project (and you!) can copy and
distribute it in the United States without permission and
without paying copyright royalties. Special rules, set forth
below, apply if you wish to copy and distribute this eBook
under the "PROJECT GUTENBERG" trademark.

Please do not use the "PROJECT GUTENBERG" trademark to market
any commercial products without permission.

To create these eBooks, the Project expends considerable
efforts to identify, transcribe and proofread public domain
works. Despite these efforts, the Project's eBooks and any
medium they may be on may contain "Defects". Among other
things, Defects may take the form of incomplete, inaccurate or
corrupt data, transcription errors, a copyright or other
intellectual property infringement, a defective or damaged
disk or other eBook medium, a computer virus, or computer
codes that damage or cannot be read by your equipment.

LIMITED WARRANTY; DISCLAIMER OF DAMAGES
But for the "Right of Replacement or Refund" described below,
[1] Michael Hart and the Foundation (and any other party you may
receive this eBook from as a PROJECT GUTENBERG-tm eBook) disclaims
all liability to you for damages, costs and expenses, including
legal fees, and [2] YOU HAVE NO REMEDIES FOR NEGLIGENCE OR
UNDER STRICT LIABILITY, OR FOR BREACH OF WARRANTY OR CONTRACT,
INCLUDING BUT NOT LIMITED TO INDIRECT, CONSEQUENTIAL, PUNITIVE
OR INCIDENTAL DAMAGES, EVEN IF YOU GIVE NOTICE OF THE
POSSIBILITY OF SUCH DAMAGES.

If you discover a Defect in this eBook within 90 days of
receiving it, you can receive a refund of the money (if any)
you paid for it by sending an explanatory note within that
time to the person you received it from. If you received it
on a physical medium, you must return it with your note, and
such person may choose to alternatively give you a replacement
copy. If you received it electronically, such person may
choose to alternatively give you a second opportunity to
receive it electronically.

THIS EBOOK IS OTHERWISE PROVIDED TO YOU "AS-IS". NO OTHER
WARRANTIES OF ANY KIND, EXPRESS OR IMPLIED, ARE MADE TO YOU AS
TO THE EBOOK OR ANY MEDIUM IT MAY BE ON, INCLUDING BUT NOT
LIMITED TO WARRANTIES OF MERCHANTABILITY OR FITNESS FOR A
PARTICULAR PURPOSE.

Some states do not allow disclaimers of implied warranties or
the exclusion or limitation of consequential damages, so the
above disclaimers and exclusions may not apply to you, and you
may have other legal rights.

INDEMNITY
You will indemnify and hold Michael Hart, the Foundation,
and its trustees and agents, and any volunteers associated
with the production and distribution of Project Gutenberg-tm
texts harmless, from all liability, cost and expense, including
legal fees, that arise directly or indirectly from any of the
following that you do or cause:  [1] distribution of this eBook,
[2] alteration, modification, or addition to the eBook,
or [3] any Defect.

DISTRIBUTION UNDER "PROJECT GUTENBERG-tm"
You may distribute copies of this eBook electronically, or by
disk, book or any other medium if you either delete this
"Small Print!" and all other references to Project Gutenberg,
or:

[1]  Only give exact copies of it.  Among other things, this
     requires that you do not remove, alter or modify the
     eBook or this "small print!" statement.  You may however,
     if you wish, distribute this eBook in machine readable
     binary, compressed, mark-up, or proprietary form,
     including any form resulting from conversion by word
     processing or hypertext software, but only so long as
     *EITHER*:

     [*]  The eBook, when displayed, is clearly readable, and
          does *not* contain characters other than those
          intended by the author of the work, although tilde
          (~), asterisk (*) and underline (_) characters may
          be used to convey punctuation intended by the
          author, and additional characters may be used to
          indicate hypertext links; OR

     [*]  The eBook may be readily converted by the reader at
          no expense into plain ASCII, EBCDIC or equivalent
          form by the program that displays the eBook (as is
          the case, for instance, with most word processors);
          OR

     [*]  You provide, or agree to also provide on request at
          no additional cost, fee or expense, a copy of the
          eBook in its original plain ASCII form (or in EBCDIC
          or other equivalent proprietary form).

[2]  Honor the eBook refund and replacement provisions of this
     "Small Print!" statement.

[3]  Pay a trademark license fee to the Foundation of 20% of the
     gross profits you derive calculated using the method you
     already use to calculate your applicable taxes.  If you
     don't derive profits, no royalty is due.  Royalties are
     payable to "Project Gutenberg Literary Archive Foundation"
     the 60 days following each date you prepare (or were
     legally required to prepare) your annual (or equivalent
     periodic) tax return.  Please contact us beforehand to
     let us know your plans and to work out the details.

WHAT IF YOU *WANT* TO SEND MONEY EVEN IF YOU DON'T HAVE TO?
Project Gutenberg is dedicated to increasing the number of
public domain and licensed works that can be freely distributed
in machine readable form.

The Project gratefully accepts contributions of money, time,
public domain materials, or royalty free copyright licenses.
Money should be paid to the:
"Project Gutenberg Literary Archive Foundation."

If you are interested in contributing scanning equipment or
software or other items, please contact Michael Hart at:
hart@pobox.com

[Portions of this eBook's header and trailer may be reprinted only
when distributed free of all fees.  Copyright (C) 2001, 2002 by
Michael S. Hart.  Project Gutenberg is a TradeMark and may not be
used in any sales of Project Gutenberg eBooks or other materials be
they hardware or software or any other related product without
express permission.]

*END THE SMALL PRINT! FOR PUBLIC DOMAIN EBOOKS*Ver.02/11/02*END*

\end{verbatim}
\end{document}
